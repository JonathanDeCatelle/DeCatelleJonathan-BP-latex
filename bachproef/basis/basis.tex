%%=============================================================================
%% Basisfunctionaliteiten
%%=============================================================================

\chapter{Basisfunctionaliteiten}%
\label{ch:basisfunctionaliteiten}

In dit hoofdstuk worden de basisfunctionaliteiten van native en cross-platform vergeleken met elkaar. 
Met de resultaten kan dan een gepaste conclusie worden gevormd.

\section{Native}
\subsubsection{Wat hebben we nodig}
Normaal gezien wordt er voor de functionaliteiten altijd een of andere library of API gebruikt. 
Enkel bij de navigatie is dit niet nodig. Voor de navigatie wordt het 
startproject dat Android Studio aanbiedt gebruikt, hierin zit de navigatie al geïmplementeerd 
\ref{par:basisfunctionaliteiten}. Voor het laadscherm dat getoond wordt bij het opstarten van 
de applicatie zal de SplashScreen API gebruikt worden die wordt aangeboden door Android Studio.

\subsubsection{Uitvoering}

Voor de navigatie is het voldoende om het nieuwe project aan te maken. 
Hierbij zit de navigatie al geïmplemnteerd.

\paragraph{1. Gradle instellingen aanpassen}
Om de SplashScreen API te gebruiken, moeten deze aan de dependancies worden toegevoegd. Dit wordt gedaan in
het \textbf{build.gradle(module)} bestand.
\begin{minted}{kotlin}
dependencies {
    // andere dependacies
    implementation("androidx.core:core-splashscreen:1.0.0")
}
\end{minted}

\paragraph{2. SplashScreen instellen}
Na het toevoegen van de dependancy kan het laadscherm aangepast worden. 
Hiervoor wordt een nieuw \textbf{splash.xml} bestand in de \textbf{res/values} map aangemaakt. In dit 
bestand kan het laadscherm gecustomized worden. Ook wordt hier de icoon toegevoegd die wordt
getoond tijdens het laden. Dit icoon wordt in de \textbf{res/drawable} map geplaatst.
\begin{minted}{xml}
<?xml version="1.0" encoding="utf-8"?>
<resources>
    <style name="Theme.MyApp.MySplash" parent="Theme.SplashScreen">
        <item name="windowSplashScreenBackground">@color/black</item>
        <item name="windowSplashScreenAnimatedIcon">
            @drawable/bank_svgrepo_com</item>
        <item name="postSplashScreenTheme">@style/Theme.Basis</item>
    </style>
</resources>
\end{minted}

\paragraph{3. Applicatie maken}
Dankzij deze informatie wordt een applicatie opgezet die een laadscherm toont dat weer zal verdwijnen 
van zodra de applicatie zijn eerste frame tekent. Daarna is er onderaan een navigatiebar te zien die ervoor zorgt dat 
er genavigeerd kan worden tussen de verschillende schermen.
\begin{figure}[H]
    \centering
    \includegraphics[height=0.4\textheight]{basis_layoutnative.png}
    \caption{Layout van applicatie voor de basisfunctionaliteiten bij Android.}
\end{figure}


\subsubsection{Ontwikkeltijd}

Dankzij het startproject dat Android Studio aanbiedt, is het mogelijk om snel een applicatie op te zetten 
die gebruik maakt van de navigatiecomponent. Daarnaast kan het laadscherm ook snel geïmplementeerd worden. 
Er moet enkel al een icon aanwezig zijn om te tonen. Het duurde ongever 30 minuten om de applicatie op te
zetten.



\subsubsection{Performantie}

\paragraph{Tijdsduur}
\begin{figure}[H]
    \centering
    \includegraphics[height=0.1\textheight]{basisDuratieNative.png}
    \caption{Overzicht tijdsduur opstarten van applicatie met basisfunctionaliteiten bij Android.}
\end{figure}
Tijdens het meten van de duur voor het opstarten van de applicatie, 
is de applicatie meerdere keren opgestart. Op de grafiek is te zien dat de applicatie
gemiddeld 2,58 seconden nodig heeft om op te starten. Het minimum en maximum 
liggen op 1,37 en 3,56 seconden.

\paragraph{CPU \& geheugen}
\begin{figure}[H]
    \centering
    \includegraphics[height=0.25\textheight]{basisPerformantieNative.png}
    \caption{Overzicht CPU en geheugen gebruik tijdens het navigeren tussen schermen bij Android.}
\end{figure}
Op de grafiek is te zien dat het CPU gebruik van de applicatie wanneer deze 
inactief is, rond de 0 - 1\% ligt en dat het duidelijk is wanneer er
genavigeerd wordt tussen de schermen. De piek van het CPU gebruik lag gemiddeld
op 31\% met een minimum en maximum van 25\% en 42\%. Het geheugen blijft in tegenstelling
met de CPU wanneer de applicatie inactief en actief is, rond de 81MB hangen, met
verschillen van maximum 2-3MB. Er is geen merkbaar verschil in het geheugen wanneer
er genavigeerd wordt tussen de schermen.



\subsubsection{Schaalbaarheid}

\paragraph{Complexiteit}
De navigatie is snel om op te zetten aangezien er een project kan worden aangemaakt met de navigatie ingebouwd.
Het is ook niet complex om daarna meerdere schermen toe te voegen. 
Het laadscherm is ook niet complex om op te zetten. Het enige dat nodig is, is een icon dat getoond zal worden.

\paragraph{Herbruikbaarheid}
De navigatie kan zoals eerder vermeld hergebruikt worden voor extra schermen toe te voegen en is dus gemakkelijk 
om uit te breiden. Het is ook gemakkelijk om de bestaande schermen te hergebruiken in andere applicaties. 
Bij het laadscherm is er geen sprake van opschaling. Dit is enkel nodig bij het opstarten 
van de applicatie en zal dus maar 1 keer geïmplementeerd moeten worden.



\section{Cross-platform}
\subsubsection{Wat hebben we nodig}
Bij React Native is er voor de navigatie een library nodig. Hiervoor wordt
React Navigation gebruikt. Voor het laadscherm wordt react-native-bootsplash library gebruikt. 
Met behulp van deze libraries kan er genavigeerd worden in de applicatie en kan er een laadscherm
getoond worden.

\subsubsection{Uitvoering}

\paragraph{1. Library toevoegen}
Om de navigatie en het laadscherm te kunnen gebruiken moeten we de juiste libraries aan 
de root van ons project toevoegen.
\begin{minted}{bash}
npm install @react-navigation/native
npm install react-native-screens
npm install react-native-safe-area-context
npm install @react-navigation/bottom-tabs
\end{minted}

\paragraph{2. onCreate methode toevoegen}
Om de navigatie te kunnen gebruiken moeten we een \textbf{onCreate} methode toevoegen aan het
\textit{android/app/src/main/java/com/project/MainActivity.java} bestand. Dit doen we door de 
volgende regels toe te voegen.
\begin{minted}{java}
import android.os.Bundle;
// ...
public class MainActivity extends ReactActivity {
  // ...
  @Override
  protected void onCreate(Bundle savedInstanceState) {
    super.onCreate(null);
  }
  // ...
}
\end{minted}

\paragraph{3. Dependencies toevoegen}
Om het laadscherm te kunnen gebruiken moeten we de juiste dependencies toevoegen aan het
\textit{android/app/build.gradle} bestand. Dit doen we door de volgende regels toe te voegen.
\begin{minted}{java}
dependencies {
  // Andere dependencies
  implementation("androidx.core:core-splashscreen:1.0.0")
}
\end{minted}

\paragraph{4. Laadscherm toevoegen}
Eerst genereren we een Laadscherm met behulp van een commando aangeboden door de library.
\begin{minted}{bash}
npx react-native generate-bootsplash path/to/icon.png
    --background-color "#000000" 
    --logo-width=100 
    --assets-path=assets 
    --flavor=main 
    --platforms=android,ios
\end{minted}
Daarna maken we een stijl om het laadschem te tonen. Dit doen we door de volgende regels toe te
voegen aan het \textit{android/app/src/main/res/values/styles.xml} bestand.
\begin{minted}{xml}
<style name="BootTheme" parent="Theme.SplashScreen">
    <item name="windowSplashScreenBackground">@color/bootsplash_background</item>
    <item name="windowSplashScreenAnimatedIcon">@mipmap/bootsplash_logo</item>
    <item name="postSplashScreenTheme">@style/AppTheme</item>
</style>
\end{minted}
Om het laadscherm dan als eerste te tonen moeten we de volgende regels toevoegen aan het
\textit{android/app/src/main/AndroidManifest.xml} bestand.
\begin{minted}{xml}
<application
  android:name=".MainApplication"
  android:label="@string/app_name"
  android:icon="@mipmap/ic_launcher"
  android:roundIcon="@mipmap/ic_launcher_round"
  android:allowBackup="false"
  android:theme="@style/BootTheme"> <!-- Verander deze lijn -->
  <!-- ... -->
</application>
\end{minted}
Tot slot initialiseren we het laadscherm door de volgende regels toe te voegen aan het
\textit{android/app/src/main/java/com/project/MainApplication.java} bestand.
\begin{minted}{java}
import com.zoontek.rnbootsplash.RNBootSplash;

public class MainApplication extends Application implements ReactApplication {
  // ...
  @Override
  public void onCreate(Bundle savedInstanceState) {
    RNBootSplash.init(this); // Voeg deze lijn toe
    super.onCreate(savedInstanceState);
  }
  // ...
}
\end{minted}

\paragraph{5. Navigatie toevoegen}
Om de navigatie te kunnen gebruiken moeten we alle alle componenten in een 
\textbf{NavigationContainer} component steken. Dit doen we door de volgende regels toe te voegen
aan het \textit{App.tsx} bestand.
\begin{minted}{typescript}
import { NavigationContainer } from '@react-navigation/native';

export default function App() {
  return (
    <NavigationContainer>
      {/* ... */}
    </NavigationContainer>
  );
}
\end{minted}

\paragraph{6. Bottom Tab navigatie toevoegen}
Om de bottom tab navigatie te kunnen gebruiken moeten we een \textbf{createBottomTabNavigator}
functie aanmaken. Dit doen we door de volgende regels toe te voegen aan het \textit{App.tsx} bestand.
\begin{minted}{typescript}
import { createBottomTabNavigator } from '@react-navigation/bottom-tabs';

const Tab = createBottomTabNavigator();

export default function App() {
  return (
    <NavigationContainer>
      <Tab.Navigator>
        {/* ... */}
      </Tab.Navigator>
    </NavigationContainer>
  );
}
\end{minted}
Daarna kunnen we nog de verschillende schermen toevoegen aan de \textbf{Tab.Navigator} component.
\begin{minted}{typescript}
<Tab.Navigator>
    <Tab.Screen name="Home" component={HomeScreen} />
    <Tab.Screen name="Second" component={SecondScreen} />
    <Tab.Screen name="Third" component={ThirdScreen} />
</Tab.Navigator>
\end{minted}

\paragraph{7. Applicatie maken}
Met deze informatie maken we een applicatie met een laadscherm dat zal verdwijnen vanaf de 
applicatie gerenderd is. Daarna is er een bottom tab navigatie dat navigeert tussen 
de drie verschillende schermen. 
\begin{figure}[H]
    \centering
    \includegraphics[height=0.4\textheight]{basis_layoutcross.png}
    \caption{Layout van applicatie voor de basisfunctionaliteiten bij React native.}
\end{figure}

\subsubsection{Ontwikkeltijd}

Het implementeren van beide libraries is niet moeilijk. Voor de navigatie te implementeren 
is ongeveer 30 minuten nodig. Voor het laadscherm is ook ongeveer 20 minuten nodig. Dus in totaal 
ongeveer 50 minuten om beide libraries te implementeren. Het is wel belangrijk 
om aandachtig de installatiestappen te volgen. Indien dit niet gebeurt kunnen er makkelijk fouten optreden.
Waardoor tijd verloren gaat aan het oplossen van deze fouten.




\subsubsection{Performantie}

\paragraph{Tijdsduur}
\begin{figure}[H]
    \centering
    \includegraphics[height=0.1\textheight]{basisDuratieCross.png}
    \caption{Overzicht tijdsduur opstarten van applicatie met basisfunctionaliteiten bij React Native.}
\end{figure}
Tijdens het meten van de duur voor het opstarten van de applicatie, 
is de applicatie meerdere keren opgestart. Op de grafiek is te zien dat de applicatie
gemiddeld 2,84 seconden nodig heeft om op te starten. Het minimum en maximum
liggen op 2,29 en 6,59 seconden.

\paragraph{CPU \& geheugen}
\begin{figure}[H]
    \centering
    \includegraphics[height=0.25\textheight]{basisPerformantieCross.png}
    \caption{Overzicht CPU en geheugen gebruik tijdens het navigeren tussen schermen bij React Native.}
\end{figure}
Net zoals bij native is op de grafiek te zien dat het CPU gebruik van de applicatie wanneer deze
inactief is, rond de 4\% ligt. En dat er duidelijk zichtbaar is wanneer er
genavigeerd wordt tussen de schermen. De piek van het CPU gebruik lag gemiddeld
op 32\% met een minimum en maximum van 27\% en 40\%. Wat bij React Native opvalt is dat
het eerste keer navigeren naar een scherm een grotere piek geeft. Eenmaal dat het scherm 
al werd ingeladen is de piek minder groot. Dit wijst erop dat React Native gebruik maakt van
lazy loading. Het geheugen blijft in tegenstelling
met de CPU wanneer de applicatie inactief en actief is, rond de 202MB hangen, met
verschillen van maximum 4-5MB. Er is geen merkbaar verschil in het geheugen wanneer
er genavigeerd wordt tussen de schermen.


\subsubsection{Schaalbaarheid}

\paragraph{Complexiteit}
Het gebruik van de libraries is niet complex. De navigatie is snel om op te zetten dankzij de documentatie van de library.
Het laadscherm is ook niet complex om op te zetten. Het enige dat nodig is, is een icon dat getoond zal worden en de 
initiële setup van de library. 

\paragraph{Herbruikbaarheid}
Het toevoegen van extra schermen is niet complex en kan dus makkelijk hergebruikt worden. Het is ook makkelijk om de
bestaande schermen te herbruiken in andere applicaties. Bij het laadscherm is er geen sprake van opschaling. Dit is enkel
nodig bij het opstarten van de applicatie en zal dus maar 1 keer geïmplementeerd moeten worden.



\section{Conclusie}
Op basis van de verzamelde gegevens blijkt dat native ontwikkeling een betere
performantie heeft dan cross-platform ontwikkeling voor de basisfunctionaliteiten. De gemiddelde 
opstarttijd van de native applicatie is sneller, met een gemiddelde van 2,58 seconden 
in vergelijking met 2,84 seconden bij cross-platform. Daarnaast heeft de native 
applicatie een lager gemiddeld geheugengebruik van ongeveer 81MB, terwijl de cross-platform 
applicatie gemiddeld 202MB aan geheugen vereist. Desondanks beide applicaties
vergelijkbaar CPU-gebruik hebben, toont dit onderzoek aan dat native applicaties voor
basisfunctionaliteiten beter zijn op vlak van opstartsnelheid en geheugenverbruik.
\\\\
\begin{tabular}{ |p{3cm}||p{6cm}|p{6cm}| }
    \hline
     & Native (Android) & Cross-platform (React Native) \\
    \hline
     & \multicolumn{2}{|c|}{Tijdsduur} \\
    \hline
    Minimaal & 1,37s & 2,29s \\
    Maximaal & 3,56s & 6,59s \\
    Gemiddeld & 2,58s & 2,84s \\
    \hline
     & \multicolumn{2}{|c|}{Geheugen} \\ 
    \hline
    Offset & 2-3MB & 4-5MB \\
    Gemiddeld & 81MB & 202MB \\
    \hline
     & \multicolumn{2}{|c|}{CPU} \\
    \hline
    Minimaal & 25\% & 27\% \\
    Maximaal & 42\% & 40\% \\
    Gemiddeld & 31\% & 32\% \\
    \hline
\end{tabular}
\\\\
Met het behulp van Android Studio en het startproject dat Android Studio aanbied, 
kunnen de basisfunctionaliteiten sneller worden geïmplementeerd bij 
native applicaties dan bij cross-patform. Dit komt omdat installatieproces van 
de gebruikte libraries bij cross-platform meer werk vraagt dan bij native applicaties. 
Daarnaast moet de naviagtie bij cross-platform zelf worden opgebouwd wat opnieuw meer tijd vraagt.
\\\\
Op het gebied van schaalbaarheid zijn beide ontwikkelmethodes een goede keuze aangezien 
het bij beide ontwikkelmethodes gemakkelijk is om de navigatiecomponenten uit te breiden met extra schermen. 
Daarnaast is het ook gemakkelijk om bestaande schermen te hergebruiken.
\\\\
Op vlak van performantie gaat de voorkeur naar native. Daarnaast als er rekening wordt gehouden met het feit dat 
native enkel Android bevat gaat de voorkeur op vlak van ontwikkeltijd naar cross-platform.
Op het vlak van schaalbaarheid is er geen voorkeur tussen beide ontwikkelmethodes.




















