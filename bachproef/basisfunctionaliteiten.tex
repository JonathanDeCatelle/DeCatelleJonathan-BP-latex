%%=============================================================================
%% Basisfunctionaliteiten
%%=============================================================================

\chapter{Basisfunctionaliteiten}%
\label{ch:basisfunctionaliteiten}

In dit hoofdstuk gaan we de basisfunctionaliteiten van native en cross-platform vergelijken. 
Met deze resultaten kunnen we dan een gepaste conclusie vormen.

\section{Native}
\subsubsection{Wat hebben we nodig}
%tools, libraries, ...
Vooraleer we kunnen beginnen met het implementeren van de functionaliteit moeten we eerst kijken naar hoe we dit best kunnen doen. 
Gelukkig bied Android Studio een template aan als startproject waarin de basisfunctionaliteit van navigeren tussen scherm ingewerkt 
zit \ref{par:basisfunctionaliteiten}. Daarnaast moeten we wel zoeken naar een manier om een laadscherm te tonen terwijl de applicatie 
aan het opstarten is.

\subsubsection{Uitvoering}

\subsubsection{Ontwikkeltijd}

\paragraph{Geïnvesteerde tijd}

\paragraph{Compiletijd}

\subsubsection{Performantie}

\subsubsection{Schaalbaarheid}

\subsubsection{Conclusie}


\section{Cross-platform}
\subsubsection{Wat hebben we nodig}
%tools, libraries, ...

\subsubsection{Uitvoering}

\subsubsection{Ontwikkeltijd}

\paragraph{Geïnvesteerde tijd}

\paragraph{Compiletijd}

\subsubsection{Performantie}

\subsubsection{Schaalbaarheid}

\subsubsection{Conclusie}






















