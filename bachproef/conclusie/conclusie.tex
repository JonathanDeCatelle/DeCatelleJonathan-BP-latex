%%=============================================================================
%% Conclusie
%%=============================================================================

\chapter{Conclusie}%
\label{ch:conclusie}

\section{Hoofdonderzoeksvraag}
In dit onderzoek werd er gezocht naar een antwoord op de hoofdonderzoeksvraag:
\begin{itemize}
    \item Hoe verschillen de prestaties, schaalbaarheid en ontwikkeltijd van functionaliteiten tussen native en cross-platform ontwikkeling van mobiele applicaties?
\end{itemize}
Om een antwoord te vinden op deze vraag werden er twee applicaties ontwikkeld, één native en één cross-platform.
Bij deze applicaties werden dan verschillende functionaliteiten geïmplementeerd en is de performantie getest. 
Daarnaast is er ook gekeken naar de ontwikkeltijd en schaalbaarheid van de applicaties. 

\subsection{Ontwikkeltijd}
\subsubsection{Algemene ontwikkeltijd}
Over het algemeen is de ontwikkeltijd bij cross-platform langer dan bij native.
Dit komt door de libraries die geïnstalleerd moeten worden, wat extra tijd in beslag neemt en meer 
ruimte geeft voor fouten. Deze fouten moeten dan ook nog eens opgelost worden, wat ook weer extra tijd in beslag neemt.
\\\\
De ontwikkeltijd is ook langer omdat de React Native libraries de bestaande implementatie van de 
native platformen gebruiken en op verder bouwen. Hierdoor is er soms meer code nodig bij 
cross-platform om dezelfde functionaliteit te krijgen.

\subsubsection{Compiletijd}
De compiletijd bij cross-platform is altijd langer dan bij native. Dit komt omdat er bij cross-platform
een extra stap is bij het compileren. De code wordt eerst gecompileerd naar Javascript, dan pas naar native.
Bij native is er maar één stap, de code wordt gecompileerd naar native.
\\\\
Ondanks het feit dat de compiletijd langer is bij cross-platform, is het niet altijd nodig om de applicatie te compileren.
React Native heeft een hot reload functie, waardoor de applicatie niet gecompileerd moet worden bij het testen.
Eenmaal de applicatie gecompileerd is, kunnen er veranderingen worden aangebracht aan de code en deze worden dan
automatisch doorgevoerd in de applicatie. Dit is een groot voordeel bij het ontwikkelen van de applicatie,
omdat er geen tijd verloren gaat aan het compileren van de applicatie.

\subsection{Performantie}
\subsubsection{Tijdsduur}
Op het vlak van tijdsduur is er enkel een groot verschil bij het aanmaken van notificaties.
Bij cross-platform duurt het aanmaken van een notificatie 14 keer langer dan bij native.
Ook al is dit een groot verschil, het aanmaken van een notificatie duurt bij cross-platform 
nog steeds maar 166ms. Dit is nog steeds snel genoeg om de gebruiker niet te storen 
aangezien dit gebeurt op de achtergrond. Bij de andere functionaliteiten is er wel een verschil
in tijdsduur, maar dit verschil is niet zo groot dat het storend is voor de gebruiker of dat het een 
reden is om cross-platform niet te gebruiken.

\subsubsection{CPU gebruik}
Bij het CPU gebruik zijn de resultaten wisselvallig. Over het algemeen is het CPU gebruik bij cross-platform
hoger dan bij native, maar dit is niet altijd het geval. Bij het gebruik van sensoren bijvoorbeeld is het CPU gebruik
bij cross-platform lager dan bij native. Maar als we dan kijken naar het CPU gebruik bij het afspelen van audio en 
video, dan is het CPU gebruik bij cross-platform hoger dan bij native. Bij het kiezen van een ontwikkelmethode is het 
dus belangrijk om te kijken naar de functionaliteiten die gebruikt zullen worden in de applicatie. 
Deze zullen bepalend zijn voor de uiteindelijke keuze van de ontwikkelmethode.

\subsubsection{Geheugengebruik}
Op het vlak van geheugengebruik is nooit een merkbaar verschil geweest bij het uitvoeren van functionaliteiten 
of wanneer de applicatie inactief is. Daarnaast was het gemiddelde geheugengebruik bij alle functionaliteiten 
bij cross-platform veel hoger dan bij native. Het geheugengebruik bij cross-platform is gemiddeld 2,5 keer hoger
dan bij native. Dit is een groot verschil, als gevolg van de libraries die gebruikt worden bij cross-platform.
Alle extra libraries die gebruikt worden bij cross-platform nemen extra geheugen in beslag. Ook de React 
Native library die altijd standaard is geïnstalleerd, neemt extra geheugen in beslag. Dit is een nadeel bij
cross-platform, maar het is niet zo groot dat het een reden is om cross-platform niet te gebruiken aangezien 
de meeste smartphones tegenwoordig genoeg geheugen hebben om cross-platform applicatie te runnen zonder dat
de gebruiker hier iets van merkt.

\subsection{Schaalbaarheid}
\subsubsection{Complexiteit}
Over het algemeen is de complexiteit gelijkaardig bij native en cross-platform. Bij cross-platform is er wel
meer complexiteit bij het opzetten van de applicatie, maar dit is een eenmalige complexiteit. Er bestaan wel 
complexe libraries bij cross-platform, maar deze zijn niet verplicht om te gebruiken.

\subsubsection{Herbruikbaarheid}
Zowel native als cross-platform hebben een hoge herbruikbaarheid. Bij beide methodes is het mogelijk
om de code te centraliseren en te hergebruiken op andere plaatsen binnen de applicatie of vanop de 
gecentraliseerde plaats de code op te schalen. Het is ook mogelijk om de code te hergebruiken in 
andere applicaties.

\section{Deelonderzoeksvragen}
Naast de hoofdonderzoeksvraag zijn er twee deelonderzoeksvragen waar er een antwoord op gezocht werd 
bij het uitvoeren van dit onderzoek. 
\begin{itemize}
    \item Zijn er functionaliteiten die cross-platform niet ondersteunen?
    \item Zijn er functionaliteiten bij cross-platform waarbij de performantie de functionaliteit onbruikbaar maakt?
\end{itemize}
Bij de vraag of er functionaliteiten zijn die cross-platform niet ondersteunt, 
is er gebleken dat dit niet het geval is. Alle functionaliteiten die in dit onderzoek zijn geïmplementeerd
waren mogelijk bij cross-platform. Bij sommige functionaliteiten had cross-platform meerdere libraries 
die de functionaliteit ondersteunden, terwijl er bij native altijd maar één implementatie mogelijk was.
\\\\
Bij de vraag of er functionaliteiten onbruikbaar zijn bij cross-platform ontwikkeling vanwege de performantie, 
is de conclusie getrokken dat dit niet het geval is. Alle functionaliteiten die in dit onderzoek zijn geïmplementeerd,
waren bruikbaar bij cross-platform. Er was wel altijd een verschil in performantie tussen native en cross-platform,
maar dit verschil was nooit zo groot dat de functionaliteit onbruikbaar was.

\section{Wat nu?}
Op vlak van performantie is er duidelijk een verschil tussen native en cross-platform. 
Dit roept de vraag op of dit bij alle functionaliteiten het geval zal zijn, 
en of er zelfs functionaliteiten zijn die beter presteren bij cross-platform. 
Het kan ook zijn dat cross-platform frameworks altijd met een achterstand zullen zitten op native frameworks.
Daarom zijn dit soort onderzoeken belangrijk, om een inzicht te krijgen in de vooruitgang van cross-platform frameworks.
\\\\
Het onderzoek kan ook uitgebreid worden naar andere cross-platform frameworks zoals Flutter of 
.NET MAUI. Het kan zijn dat deze frameworks beter presteren op vlak van performantie zonder te 
comprimeren op de ontwikkeltijd en schaalbaarheid. Het onderzoek kan ook uitgebreid worden 
naar andere native frameworks zoals Java of zelfs andere native platformen zoals IOS. 
Eigenlijk zin de mogelijkheden om dit onderzoek uit te breiden of voort te zetten zijn eindeloos.
\\\\
Het debat over native en cross-platform zal altijd blijven bestaan. Er komen steeds nieuwe frameworks
en technologieën bij die het debat weer doen oplaaien. Een antwoord op de vraag welke 
ontwikkelingsmethode/framework uiteindelijk het beste is, zal er nooit komen. Dit soort onderzoeken 
echter kan een ontwikkelaar wel helpen om een doordachte keuze te maken 
bij het kiezen van een technologie voor zijn eigen specifieke project.






