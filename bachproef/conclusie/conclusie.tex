%%=============================================================================
%% Conclusie
%%=============================================================================

\chapter{Conclusie}%
\label{ch:conclusie}

% TODO: Trek een duidelijke conclusie, in de vorm van een antwoord op de
% onderzoeksvra(a)g(en). Wat was jouw bijdrage aan het onderzoeksdomein en
% hoe biedt dit meerwaarde aan het vakgebied/doelgroep? 
% Reflecteer kritisch over het resultaat. In Engelse teksten wordt deze sectie
% ``Discussion'' genoemd. Had je deze uitkomst verwacht? Zijn er zaken die nog
% niet duidelijk zijn?
% Heeft het onderzoek geleid tot nieuwe vragen die uitnodigen tot verder 
%onderzoek?

\section{Hoofdonderzoeksvraag}
In dit onderzoek wordt een antwoord gegeven op de hoofdonderzoeksvraag 
\begin{itemize}
    \item Hoe verschillen de prestaties, schaalbaarheid en ontwikkeltijd van functionaliteiten tussen native en cross-platform ontwikkeling van mobiele applicaties?
\end{itemize}
Om deze vraag te beantwoorden is een onderzoek uitgevoerd door verschillende applicaties op te stellen. 
Bij deze applicaties werden dan verschillende functionaliteiten geïmplementeerd en is de performantie, 
schaalbaarheid en ontwikkeltijd vergeleken met elkaar. 
\\\\
Uit de resultaten van dit onderzoek is gebleken dat er...
%grafiek van de performantie per functionaliteit of als overzicht

\paragraph{Compiletijd}


\section{Deelonderzoeksvragen}
Naast de hoofdonderzoeksvraag zijn er ook aantal deelonderzoeksvragen waar er een antwoord op proberen 
geven dankzij het uitvoeren van dit onderzoek. 
Deze deelonderzoeksvragen waren volgende.
\begin{itemize}
    \item Zijn er functionaliteiten die cross-platform niet ondersteunen?
    \item Zijn er functionaliteiten waarvan de performantie bij cross-platform het onbruikbaar maakt?
\end{itemize}
Dankzij het uitgevoerde onderzoek kunnen we hierop antwoorden. Bij de vraag of dat er 
functionaliteiten zijn die cross-platform niet ondersteund, 
is er gebleken dat...

Bij de vraag of dat er functionaliteiten onbruikbaar zijn bij cross-platform dankzij de performantie 
zijn we tot de conclusie gekomen dat dit...

\section{Wat nu?}
%Zoals we uit de conclusies kunnen zien is er een duidelijk verschil tussen de performantie, schaalbaarheid en 
%ontwikkeltijd van functionaliteiten bij native en cross-platform. 

%Hierdoor wordt er afgevraagd of dat bij alle functionaliteiten zo zal zijn, of dat er wel functionaliteiten zijn met 
%een betere performantie bij cross-platform. 





