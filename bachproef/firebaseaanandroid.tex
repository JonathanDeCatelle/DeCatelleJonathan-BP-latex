\subparagraph{1. Configuratie bestanden toevoegen}
In het Firebase project voegen we een nieuwe Android applicatie toe. Om deze aan te maken geven we de package naam 
dat te vinden is in build.gradle(module) bestand als applicationId en optioneel de naam van de applicatie mee.
\\\\
Na dat de Android applicatie is aangemaakt, moeten we het \textbf{google-services.json} bestand downloaden. 
Deze plaatsen we dan in de \textit{app} folder.

\subparagraph{2. Firebase plugins configureren}
Om Firebase het configuratiebestand nu te laten gebruiken moeten we de google-services plugin toevoegen aan 
de dependancies in het build.gradle(project) bestand boven de plugins. 
\begin{minted}{java}
buildscript {
    repositories {
        google()
        mavenCentral()
    }

    dependencies {
        // andere dependancies
        classpath "com.google.gms:google-services:4.3.15"
    }
}
\end{minted}
Daarna moeten we de plugin uitvoeren door deze aan het build.gradle(module) bestand toe te voegen.
\begin{minted}{java}
plugins {
    // andere plugins 
    id "com.google.gms.google-services"
}
\end{minted}
Tot slot moeten we de Firebase SDKs toevoegen aan het build.gradle(module) bestand.
\begin{minted}{java}
dependencies {
    // andere dependancies
    implementation platform("com.google.firebase:firebase-bom:32.0.0")
}
\end{minted}
Firebase is nu volledig aan ons project toegevoegd.