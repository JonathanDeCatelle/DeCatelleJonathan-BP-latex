\subparagraph{1. Performance Monitoring SDK aan applicatie toevoegen}
Als eerst voegen we de Performance Monitor SDK toe aan ons project in het build.gradle(module) bestand.
\begin{minted}{java}
dependencies {
    // ... andere dependancies
    implementation "com.google.firebase:firebase-perf-ktx"
}
\end{minted}

\subparagraph{2. Performance Monitoring Gradle plugin toevoegen}
Na het toevoegen van de SDK moeten we de Perormance Monitoring Gradle plugin toevoegen. 
Dit doen we door de dependancies hiervan toe te voegen aan het build.gradle(project) bestand.
\begin{minted}{java}
dependencies {
    // ... andere dependancies
    classpath "com.google.firebase:perf-plugin:1.4.2"
}
\end{minted}
Tot slot voegen we de plugin toe. Dit doen we door deze in het build.gradle(module) bestand toe te voegen.
\begin{minted}{java}
plugins {
    // ... andere plugins
    id "com.google.firebase.firebase-perf"
}
\end{minted}
Nu is het blanco project klaar om vanuit dit project alle functionaliteiten te 
implementeren en onderzoeken, buiten de basisfunctionaliteiten. Hiervoor zullen we vanuit een ander project starten, 
maar zullen we de Perormance Monitoring tool op dezelfde manier implementeren.