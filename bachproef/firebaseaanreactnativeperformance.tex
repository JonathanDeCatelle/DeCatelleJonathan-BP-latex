\subparagraph{1. Dependancy installeren}
Als eerst moeten we de Performance Monitoring tool aan de root van ons project toevoegen. 
Dit kunnen we doen met volgend commando.
\begin{minted}{bash}
npm install --save @react-native-firebase/perf
\end{minted}

\subparagraph{2. Performance Monitoring tool configureren}
Daarnaast moeten we net zoals bij de React Native Firebase app module de plugin toevoegen aan 
de dependancies binnen het \textit{/android/build.gradle} bestand. 
\begin{minted}{java}
buildscript {
    dependencies {
        // ... andere dependencies
        classpath 'com.google.firebase:perf-plugin:1.4.2'
    }
}
\end{minted}
En moeten we de plugin uitvoeren door deze aan het \textit{/android/app/build.gradle} bestand toe te voegen.
\begin{minted}{java}
apply plugin: 'com.google.firebase.firebase-perf'
\end{minted}
Nu is het blanco project klaar om vanuit dit project alle functionaliteiten te 
implementeren en onderzoeken.