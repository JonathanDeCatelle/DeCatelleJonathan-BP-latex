%%=============================================================================
%% Inleiding
%%=============================================================================

\chapter{Inleiding}
\label{ch:inleiding}

Bij het ontwikkelen van een mobiele applicatie is de keuze tussen native of cross-platform ontwikkeling altijd een belangrijke. 
Aangezien de gekozen ontwikkelingsmethode veel invloed heeft op het ontwikkelingsproces. 
Als ze native ontwikkelen zouden ze twee teams nodig hebben, één voor elk platform of één team dat de vaardigheden en tijd beschikt voor beide platformen. 
Daarnaast moet de klant over een groot genoeg budget beschikken om eventueel native te kiezen als ontwikkelingsmethode, 
aangezien native een hogere kostprijs met zich meebrengt in tegenstelling tot cross-platform.
\\\\
Beide ontwikkelingsmethodes hebben hun voor- en nadelen, en de gekozen ontwikkelingsmethode hangt af van deze voor- en nadelen. 
Native ontwikkeling wordt vaak gebruikt wanneer performantie cruciaal is, 
omdat het platform-specifieke code en speciaal ontworpen frameworks gebruikt om de applicatie te runnen. 
In vergelijking met cross-platform ontwikkeling dat gebruik maakt van een "write once, run anywhere" principe. 
Ontwikkelaars schrijven één applicatie die op IOS en Android werkt. 
Daarnaast wordt cross-platform vaak gebruikt voor simpele, niet-grafisch intensieve applicaties of bij applicaties met een hoge tijdsdruk.
\\\\
Ondanks dat native en cross-platform al vaak vergeleken zijn met elkaar wordt er nooit veel tijd gespendeerd om 
individuele functionaliteiten te vergelijken, wat in sommige gevallen belangrijk kan zijn.
Daarom gaan we in deze bachelorproef meer bepaald kijken naar de functionaliteiten 
die mobiele applicaties kunnen bevatten en de performantie, schaalbaarheid en ontwikkelingstijd ervan. 
Op die manier kunnen we een conclusie trekken uit de resultaten om daarna op basis van de gewenste functionaliteiten een ontwikkelingsmethode te kiezen.


\section{Probleemstelling}%
\label{sec:probleemstelling}

Voor veel bedrijven en ontwikkelaars is de beslissing tussen native of cross-platform om mobiele applicaties te ontwikkelen een moeilijke keuze. 
Er moeten verschillende factoren in overweging worden genomen, zoals de tijd \& budget, performantie, schaalbaarheid en de functionaliteiten. 
Indien de verkeerde keuze wordt gemaakt, kan dit een project maken of kraken. Bij native ontwikkelen kan een project snel veel tijd en geld kosten. 
Cross-platform daarentegen is een snellere en goedkopere oplossing en kan daardoor een betere keuze zijn.

\section{Onderzoeksvragen}%
\label{sec:onderzoeksvraag}

\subsection{Hoofdonderzoeksvraag}
\begin{itemize}
    \item Hoe verschillen de prestaties, schaalbaarheid en ontwikkeltijd van functionaliteiten tussen native en cross-platform ontwikkeling van mobiele applicaties?
\end{itemize}

\label{disclaimer:ios}
\textbf{\textit{Disclaimer: Bij deze bachelorproef wordt alleen gebruikgemaakt van Android langs de native kant om te vergelijken met cross-platform ontwikkeling. 
Het onderzoeken van IOS is niet mogelijk omdat dit een Apple laptop of desktop vereist, waarover ik niet beschik. 
Graag wil ik benadrukken dat het onderzoek hierdoor beperkt is tot Android en dat er geen resultaten voor IOS zijn.}}
\\\\
Om deze vraag te beantwoorden zullen we alle te onderzoeken functionaliteiten vergelijken op basis van hun performantie, 
schaalbaarheid en ontwikkeltijd gebruikmakend van native en cross-platform ontwikkelingsmethodes. 
Voor de performantie zullen we kijken naar de resources dat een functionaliteit in beslag neemt en de frames per second. 
Voor de schaalbaarheid zullen we eerst kijken of dat de functionaliteiten wel schaalbaar zijn. 
Indien ze schaalbaar zijn zullen we kijken hoe dit precies zou gebeuren om een functionaliteit op te schalen. 
Voor de ontwikkeltijd zullen we in het groot kijken naar hoeveel uren werk nodig is om een functionaliteit te implementeren. 
Daarnaast zullen we ook eventuele problemen of bugs documenteren en kijken hoelang het duurt om deze op te lossen. 
Tot slot zullen we ook kijken naar de compiletijd dat de applicaties met de geïmplementeerde functionaliteiten hebben. 
Met andere woorden, hoe lang duurt het om tijdens ontwikkeling een app te bouwen en/of veranderingen te zien.

\subsection{Deelonderzoeksvragen}

Daarnaast zijn er een aantal ondersteunende deelonderzoeksvragen die bij het onderzoek horen.

\begin{itemize}
    \item Zijn er functionaliteiten die cross-platform niet ondersteunen?
    \item Zijn er functionaliteiten waarvan de performantie bij cross-platform het onbruikbaar maakt?
    \item Kan je doorheen een project wisselen van ontwikkelingsmethode?
\end{itemize}

Dankzij dit onderzoek zullen we kijken naar de bruikbaarheid van functionaliteiten. 
Of dat cross-platform alle functionaliteiten ondersteund en of dat er bepaalde functionaliteiten zijn die een applicatie kan bevatten 
die cross-platform niet ondersteund of waarbij het verschil in performantie de functionaliteit onbruikbaar maakt. 
Tot slot zullen we ook kijken naar de mogelijkheid om tijdens het ontwikkelingsproces te wisselen van ontwikkelingsmethode. 
Van native naar cross-platform of omgekeerd, van cross-platform naar native.

\section{Onderzoeksdoelstelling}%
\label{sec:onderzoeksdoelstelling}

Dit onderzoek zal applicatie-ontwikkelaars, ondernemingen en andere geïnteresseerden een beter inzicht geven in de verschillen van functionaliteiten 
bij native en cross-platform ontwikkeling. Hierdoor zullen ze beter in staat zijn om een beslissing te maken voor hun te gebruiken ontwikkelingsmethode. 
Daarnaast zal het ook meer inzicht geven in de performantie, schaalbaarheid en ontwikkeltijd van functionaliteiten. 
Het onderzoek zal ook helpen om te bepalen of een bepaalde functionaliteit bruikbaar is bij cross-platform ontwikkeling of dat de 
performantie van de functionaliteit onbruikbaar is bij cross-platform ontwikkeling of dat de functionaliteit niet bruikbaar is bij cross-platform ontwikkeling. 
Tot slot zal het ook inzicht geven in de mogelijkheid om van ontwikkelingsmethode te wisselen doorheen een project.

\section{Opzet van deze bachelorproef}%
\label{sec:opzet-bachelorproef}

% Het is gebruikelijk aan het einde van de inleiding een overzicht te
% geven van de opbouw van de rest van de tekst. Deze sectie bevat al een aanzet
% die je kan aanvullen/aanpassen in functie van je eigen tekst.

De rest van deze bachelorproef is als volgt opgebouwd:

In Hoofdstuk~\ref{ch:stand-van-zaken} wordt een overzicht gegeven van de stand van zaken binnen het onderzoeksdomein, op basis van een literatuurstudie.

In Hoofdstuk~\ref{ch:methodologie} wordt de methodologie toegelicht en worden de gebruikte onderzoekstechnieken besproken om een antwoord te kunnen formuleren op de onderzoeksvragen.

In Hoofdstuk~\ref{ch:ontwikkelomgeving} wordt uitgelegd hoe de ontwikkelomgevingen van de gebruikte IDEs en frameworks werd opgesteld.

In Hoofdstuk~\ref{ch:projecten} wordt uitgelegd hoe dat nieuwe projecten worden aangemaakt om de functionaliteiten te implementeren.

In Hoofdstuk~\ref{ch:basisfunctionaliteiten} wordt het onderzoek naar basisfunctionaliteiten uitgevoerd.

In Hoofdstuk~\ref{ch:camera} wordt het onderzoek naar camera-integratie uitgevoerd.

In Hoofdstuk~\ref{ch:sensoren} wordt het onderzoek naar gebruik van sensoren uitgevoerd.

In Hoofdstuk~\ref{ch:notificaties} wordt het onderzoek naar push-notificaties uitgevoerd.

In Hoofdstuk~\ref{ch:audioenvideo} wordt het onderzoek naar audio- en videospelers uitgevoerd.

In Hoofdstuk~\ref{ch:conclusie}, tenslotte, wordt de conclusie gegeven en een antwoord geformuleerd op de onderzoeksvragen.













