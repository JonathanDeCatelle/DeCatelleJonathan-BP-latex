%%=============================================================================
%% Inleiding
%%=============================================================================

\chapter{Inleiding}
\label{ch:inleiding}

Bij het ontwikkelen van een mobiele applicatie is de keuze tussen native of cross-platform ontwikkeling altijd een belangrijke beslissing,
aangezien de gekozen ontwikkelmethode veel invloed heeft op het ontwikkelproces. 
Indien er gekozen wordt om native te ontwikkelen zijn er twee teams nodig, één voor elk platform, of één team dat over de vaardigheden 
en tijd beschikt voor beide platformen. 
Daarnaast moet de klant over een budget beschikken dat groot genoeg is om eventueel native te kiezen als ontwikkelmethode, 
aangezien native een hogere kostprijs met zich meebrengt in tegenstelling tot cross-platform.
\\\\
Beide ontwikkelmethodes hebben hun voor- en nadelen, en de gekozen ontwikkelmethode hangt af van deze voor- en nadelen. 
Native ontwikkeling wordt vaak gebruikt wanneer performantie cruciaal is, 
omdat het platform-specifieke code en speciaal ontworpen frameworks gebruikt om de applicatie te ontwikkelen en runnen. 
In vergelijking met cross-platform ontwikkeling dat gebruik maakt van een \textit{write once, run anywhere}-principe. 
Ontwikkelaars schrijven één applicatie die op IOS en Android werkt. 
Daarnaast wordt cross-platform vaak gebruikt voor simpele, niet-grafisch intensieve applicaties of bij applicaties met een hoge tijdsdruk.
\\\\
Ondanks het feit dat native en cross-platform al vaak vergeleken zijn met elkaar wordt er nooit veel tijd gespendeerd om 
individuele functionaliteiten te vergelijken, wat in sommige gevallen belangrijk kan zijn.
Daarom wordt er in deze bachelorproef gekeken naar de functionaliteiten 
die mobiele applicaties kunnen bevatten. Met andere woorden de functionaliteiten die in deze bachelorproef worden onderzocht,
worden vergeleken op basis van hun ontwikkeltijd, performantie en schaalbaarheid tussen native en cross-platform ontwikkeling.
Op die manier kan er een conclusie getrokken worden uit de resultaten om daarna op basis van de gewenste functionaliteiten een ontwikkelmethode te kiezen.
\\\\
Om te kunnen focussen op de verschillen tussen native en cross-platform ontwikkeling en niet de verschillen tussen de native platformen zelf,
wordt het onderzoek in deze bachelorproef langs de kant van native ontwikkeling uitgevoerd met één platform. 
Tijdens de literatuurstudie zal er besloten worden welk platform het meest geschikt is om het onderzoek op uit te voeren. 


\section{Probleemstelling}%
\label{sec:probleemstelling}

Voor veel bedrijven en ontwikkelaars is de beslissing tussen native of cross-platform om mobiele applicaties te ontwikkelen een moeilijke keuze. 
Er moeten verschillende factoren in overweging worden genomen zoals tijd, budget, performantie, schaalbaarheid en functionaliteiten. 
Indien de verkeerde keuze wordt gemaakt, kan dit een project maken of kraken. Bij native ontwikkelen kan een project snel veel tijd en geld kosten. 
Cross-platform daarentegen is een snellere en goedkopere oplossing en kan daardoor een betere keuze zijn.

\section{Onderzoeksvragen}%
\label{sec:onderzoeksvraag}

\subsection{Hoofdonderzoeksvraag}
\begin{itemize}
    \item Hoe verschillen de ontwikkeltijden, prestaties en schaalbaarheid van functionaliteiten tussen native en cross-platform ontwikkeling van mobiele applicaties?
\end{itemize}
Om deze vraag te beantwoorden worden de functionaliteiten vergeleken op basis van hun ontwikkeltijd, 
performantie en schaalbaarheid gebruikmakend van native en cross-platform ontwikkelmethodes. 
Voor de performantie wordt er gekeken naar de tijdsduur die een bepaalde actie nodig heeft, het CPU en het geheugengebruik van een applicatie.
Voor de schaalbaarheid wordt er eerst gekeken of de functionaliteit wel schaalbaar is. 
Indien deze schaalbaar is, wordt er gekeken hoe dit precies zou gebeuren en hoe gemakkelijk of moeilijk het is om de functionaliteit op te schalen. 
Voor de ontwikkeltijd wordt er in het groot gekeken naar hoeveel uren werk nodig zijn om een functionaliteit te implementeren. 
Daarnaast worden ook eventuele problemen of bugs gedocumenteerd en wordt er gekeken hoelang het duurt om deze op te lossen. 
Tot slot wordt er in het algemeen gekeken naar de compiletijd die de ontwikkelmethodes nodig hebben om een applicatie te bouwen,
met andere woorden hoelang het duurt om tijdens ontwikkeling een app te bouwen en/of veranderingen te zien.

\subsection{Deelonderzoeksvragen}
Daarnaast zijn er twee ondersteunende deelonderzoeksvragen die bij het onderzoek horen.
\begin{itemize}
    \item Zijn er functionaliteiten die cross-platform niet ondersteunen?
    \item Zijn er functionaliteiten bij cross-platform waarbij de performantie de functionaliteit onbruikbaar maakt?
\end{itemize}
Dankzij deze deelonderzoeksvragen wordt er een beter beeld geschetst over mogelijke functionaliteiten die niet 
ondersteund worden of niet bruikbaar zijn bij cross-platform ontwikkeling. Het kan zijn dat cross-platform alle functionaliteiten ondersteunt,
maar het kan ook zijn dat er bepaalde functionaliteiten zijn die cross-platform niet zal ondersteunen of waarbij het verschil 
in performantie de functionaliteit onbruikbaar maakt.


\section{Onderzoeksdoelstelling}%
\label{sec:onderzoeksdoelstelling}
Dit onderzoek zal applicatie-ontwikkelaars, ondernemingen en andere geïnteresseerden een beter inzicht geven in de verschillen van functionaliteiten 
bij native en cross-platform ontwikkeling. Hierdoor zijn ze beter in staat om een beslissing te maken voor de te gebruiken ontwikkelmethode. 
Daarnaast zal het ook meer inzicht geven in de ontwikkeltijd, performantie en schaalbaarheid van functionaliteiten. 
Het onderzoek zal ook helpen bepalen of een functionaliteit ondersteund wordt/bruikbaar is bij cross-platform ontwikkeling ofwel 
onbruikbaar gemaakt wordt door de performantie van de functionaliteit.
 

\section{Opzet van deze bachelorproef}%
\label{sec:opzet-bachelorproef}
De rest van deze bachelorproef is als volgt opgebouwd:

In Hoofdstuk~\ref{ch:stand-van-zaken} wordt een overzicht gegeven van de stand van zaken binnen 
het onderzoeksdomein, op basis van een literatuurstudie.

In Hoofdstuk~\ref{ch:methodologie} wordt de methodologie toegelicht en worden de gebruikte onderzoekstechnieken 
besproken om een antwoord te kunnen formuleren op de onderzoeksvragen.

In Hoofdstuk~\ref{ch:ontwikkelomgeving} wordt uitgelegd hoe de ontwikkelomgevingen van de gebruikte IDEs en frameworks werd opgesteld.

In Hoofdstuk~\ref{ch:projecten} wordt uitgelegd hoe nieuwe projecten worden aangemaakt om de functionaliteiten te implementeren.

In Hoofdstuk~\ref{ch:basisfunctionaliteiten} wordt het onderzoek naar basisfunctionaliteiten uitgevoerd.

In Hoofdstuk~\ref{ch:audioenvideo} wordt het onderzoek naar audio- en videospelers uitgevoerd.

In Hoofdstuk~\ref{ch:sensoren} wordt het onderzoek naar gebruik van sensoren uitgevoerd.

In Hoofdstuk~\ref{ch:notificaties} wordt het onderzoek naar push-notificaties uitgevoerd.

In Hoofdstuk~\ref{ch:conclusie}, tenslotte, wordt de conclusie gegeven en een antwoord geformuleerd op de onderzoeksvragen.













