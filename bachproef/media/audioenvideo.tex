%%=============================================================================
%% Audio- en videospelers
%%=============================================================================

\chapter{Audio- en videospelers}%
\label{ch:audioenvideo}

In dit hoofdstuk worden de audio- en video mogelijkheden van native en cross-platform vergeleken met elkaar. 
Met de resultaten kan dan een gepaste conclusie worden gevormd.

\section{Native}
\subsubsection{Wat hebben we nodig}
Om audio en video af te spelen binnen een applicatie wordt gebruik gemaakt van het MediaPlayer 
framework en de VideoView klasse. Deze stellen ons in staat om gemakkelijk audio en video af te spelen binnen 
onze applicatie. Er kunnen zowel lokale files of streams van het internet worden gebruikt.

\subsubsection{Uitvoering}

\paragraph{1. Permissions toevoegen}
Indien de applicatie ervoor moet zorgen dat het scherm niet uitvalt als er 
audio of video afspeelt, dan voegen we volgende permission toe aan het 
\textbf{AndroidManifest.xml} bestand.
\begin{minted}{xml}
<uses-permission android:name="android.permission.WAKE_LOCK" />
\end{minted}

\paragraph{2. Variabelen initialiseren, vinden en linken}
Daarna maken we een variabele aan voor de MediaPlayer en de VideoView. Deze 
variabelen worden dan gelinkt aan de juiste elementen in de layout.
\begin{minted}{kotlin}
private var mediaPlayer: MediaPlayer? = null
private var videoView: VideoView? = null

override fun onCreate(savedInstanceState: Bundle?) {
    super.onCreate(savedInstanceState)
    setContentView(R.layout.activity_main)
    // R.raw.sample is een audio file in de raw folder
    mediaPlayer = MediaPlayer.create(this, R.raw.sample)
    videoView = findViewById(R.id.videoView)
    videoView?.setVideoPath("Video URL")
}
\end{minted}

\paragraph{3. Audio en video afspelen, pauzeren en stoppen}
Om audio en video af te spelen, pauzeren en stoppen maken we gebruik van de
\textbf{start()}, \textbf{pause()} en \textbf{stop()} methodes van de
MediaPlayer en de VideoView.
\begin{minted}{kotlin}
// Audio
mediaPlayer?.start()
mediaPlayer?.pause()
mediaPlayer?.stop()

// Video
videoView?.start()
videoView?.pause()
videoView?.stopPlayback()
\end{minted}
Deze kunnen we dan koppelen aan een knoppen in de layout om de audio en video
af te spelen, pauzeren en stoppen.
\begin{minted}{kotlin}
val playButton = findViewById<Button>(R.id.playButton)
playButton.setOnClickListener {
    mediaPlayer?.start()
    videoView?.start()
}
\end{minted}

\paragraph{4. Applicatie maken}
Nu kunnen we een applicatie maken die audio en video afspeelt, pauzeert en
stopt. Deze bestaat uit een layout met telkens drie \textbf{Button} componenten 
voor het starten, pauzeren en stoppen voor zowel de audio en video. De video 
wordt afgespeeld in een \textbf{VideoView} component. De audio wordt afgespeeld
met een \textbf{MediaPlayer} object.
\begin{figure}[H]
    \centering
    \includegraphics[height=0.5\textheight]{media_layoutnative.png}
    \caption{Layout van applicatie voor het afspelen van audio en video bij Android.}
\end{figure}






\subsubsection{Ontwikkeltijd}

Er moest enkel een variabel aangemaakt worden, deze linken aan de juiste elementen
in de layout en dan de juiste methodes oproepen. Dit was zeer simpel en snel te
implementeren. De applicatie was daardoor ook zeer snel gemaakt. Het duurde ongeveer 45 minuten 
om de functionaliteiten te implementeren en de applicatie te maken.




\subsubsection{Performantie}

\paragraph{Tijdsduur}
Bij het afspelen van audio of video is het niet mogelijk om een meting te doen. 
De audio of video die afspeelt is ofwel spelend ofwel niet spelend.
Daarom wordt er enkel gekeken naar het CPU en geheugen gebruik tijdens 
het afspelen van audio en video.

\paragraph{CPU \& geheugen}
\begin{figure}[H]
    \centering
    \includegraphics[height=0.25\textheight]{mediaPerformantieNative.png}
    \caption{Overzicht CPU en geheugen gebruik tijdens het afspelen van audio en video bij Android.}
\end{figure}
De eerste twee klik events zijn voor het starten en pauzeren van de video. Op de grafiek 
is te zien dat het CPU gebruik stijgt tot 17\% en daarna afzwakt naar een wisselend gebruik 
van 0 - 7\%. Bij het pauzeren van de video stijgt het CPU gebruik tot 16\% en daarna
zakt het terug naar 0\%. Het geheugen gebruik tijdens het afspelen van de video komt overeen 
met het geheugen gebruik bij het inactief zijn van de applicatie. Er is dus geen verschil in
geheugen gebruik bij het afspelen van de video.
\\\\
De laatste twee klik events zijn voor het starten en pauzeren van de audio. Op de grafiek
is te zien dat het CPU gebruik opnieuw stijgt tot 17\%. In tegenstelling tot de video is 
er geen wisselend CPU gebruik. Het CPU gebruik blijft constant rond 0 - 1\%. Bij het pauzeren
van de audio stijgt het CPU gebruik tot 23\% en zakt daarna terug naar 0\%. Net zoals bij de 
video is er geen verschil in geheugen gebruik bij het afspelen van de audio. Het geheugen gebruik
tijdens het afspelen van de audio komt opnieuw overeen met het geheugen gebruik bij het inactief zijn van
de applicatie.


\subsubsection{Schaalbaarheid}

\paragraph{Complexiteit}
Het implementeren van een audio en video speler in Android is zeer simpel. 
Er moet geen extra library of tool worden gebruikt. Er moet enkel een variabel worden aangemaakt
en gelinkt worden aan de juiste elementen in de layout.

\paragraph{Herbruikbaarheid}
Aangezien de code zeer simpel is en er geen extra library of tool wordt gebruikt,
is de code zeer herbruikbaar. De code kan gebruikt worden in elke applicatie of klasse die audio en video
moet afspelen. In verband met de schaling, is mogelijk om de audio en video speler te abstracteren 
en deze dan vanuit één centrale plaats op te schalen. 



\section{Cross-platform}
\subsubsection{Wat hebben we nodig}
Om audio en video af te spelen binnen een cross-platform applicatie wordt gebruik gemaakt van de
react-native-track-player en react-native-video libraries. Deze stellen ons in staat om gemakkelijk 
audio en video af te spelen binnen onze applicatie. Er kunnen zowel lokale files of streams van 
het internet worden gebruikt.

\subsubsection{Uitvoering}

\paragraph{1. Library toevoegen}
Eerst moetenen de libraries aan de root van ons project worden toegevoegd. 
Deze worden toegevoegd met volgende commandos:
\begin{minted}{bash}
npm install --save react-native-track-player
npm install --save react-native-video
\end{minted}

\paragraph{2. Package teruggeven}
Normaal gezien moet de package dan worden toegevoegd aan het 
\textit{android/app/src/} \textit{main/java/com/project/MainApplication.java} bestand.
Maar dit is niet meer nodig bij React Native 0.60+.

\paragraph{3. Audio library initialiseren}
Om de audio speler te kunnen gebruiken moet een service bestand worden geregistreerd in de main 
component van de applicatie, meestal is dit het \textit{index.js} bestand.
\begin{minted}{javascript}
import TrackPlayer from 'react-native-track-player';
// AppRegistry.registerComponent(...);
TrackPlayer.registerPlaybackService(() => require('./service.js'));
\end{minted}
Daarna moet er een service worden aangemaakt in een apart \textit{service.js} bestand.
\begin{minted}{javascript}
module.exports = async function() {
    // ...
}
\end{minted}
Tot slot moet de TrackPlayer worden opgezet met de \textbf{setupPlayer()} methode. Deze methode
moet worden aangeroepen voor de \textbf{TrackPlayer} kan worden gebruikt.
\begin{minted}{javascript}
import TrackPlayer from 'react-native-track-player';

await TrackPlayer.setupPlayer()
\end{minted}

\paragraph{4. Audio en video speler gebruiken}
\subparagraph{4.1. Audio}
Om de audio speler te gebruiken moet ereerst een \textbf{Track} object worden aangemaakt.
\begin{minted}{javascript}
const track = {
    url: require(''),
    title: 'Title',
    artist: 'Artist',
    artwork: require(''),
    duration: 100
};
\end{minted}
Daarna wordt deze aan het \textbf{TrackPlayer} object toegevoegd.
\begin{minted}{javascript}
await TrackPlayer.add(track);
\end{minted}
Nu is het mogelijk om de audio te starten, pauzeren en stoppen met behulp 
van een aantal \textbf{<Button>} componenten.
\begin{minted}{javascript}
<Button title="Play" onPress={() => TrackPlayer.play()} />
<Button title="Pause" onPress={() => TrackPlayer.pause()} />
<Button title="Stop" onPress={() => TrackPlayer.stop()} />
\end{minted}

\subparagraph{4.2. Video}
Om de video speler te gebruiken moet eerst een \textbf{<Video>} component worden aangemaakt.
\begin{minted}{javascript}
<Video source={require('')} />
\end{minted}
Nu is het mogelijk om de video te starten, pauzeren en stoppen met behulp
van een aantal \textbf{<Button>} componenten.
\begin{minted}{javascript}
<Button title="Play" onPress={() => this.video.play()} />
<Button title="Pause" onPress={() => this.video.pause()} />
<Button title="Stop" onPress={() => this.video.stop()} />
\end{minted}

\paragraph{5. Applicatie maken}
De applicatie bestaat net zoals bij native voor elk de audio en video speler uit een \textbf{<View>}
component met daarin de \textbf{<Button>} componenten voor de audio of video te starten, te pauzeren
of te stoppen. Daarnaast bevat de applicatie ook een \textbf{<View>} component met daarin een
\textbf{<Video>} component voor het afspelen van de video.
\begin{figure}[H]
    \centering
    \includegraphics[height=0.4\textheight]{media_layoutcross.png}
    \caption{Layout van applicatie voor het afspelen van audio en video bij React Native.}
\end{figure}




\subsubsection{Ontwikkeltijd}

Aangezien dat de extra library react-native-track-player gebruikt wordt, wat een zeer 
uitgebreide en en soms wel complexe library is, is de ontwikkelingstijd hoger dan bij native. 
Ook moeten er meer stappen ondernomen worden om de TrackPlayer werkend te krijgen.
De extra complexiteit van react-native-track-player kan ook een voordeel zijn. Want indien de
extra functionaliteit van react-native-track-player nodig is, kan deze makkelijk geïmplementeerd worden.
Terwijl dit bij andere libraries of bij native niet altijd even makkelijk is.
\\\\
Voor videos af te spelen moesten we dan weer een andere library implementeren, wat ook weet 
tijd in beslag neemt. Maar deze library is wel makkelijk te implementeren en te gebruiken.
In totaal heeft het ongeveer 2 uur geduurd om de audio en video speler te implementeren.



\subsubsection{Performantie}

\paragraph{Tijdsduur}
Net zoals bij native wordt er enkel naar het CPU en geheugen gebruik tijdens 
het afspelen van audio en video gekeken.

\paragraph{CPU \& geheugen}
\begin{figure}[H]
    \centering
    \includegraphics[height=0.25\textheight]{mediaPerformantieCross.png}
    \caption{Overzicht CPU en geheugen gebruik tijdens het afspelen van audio en video bij React Native.}
\end{figure}
Bij het eerste klik event voor het starten van de audio stijgt het
CPU gebruik tot 42\% en zakt daarna af tot een wisselend gebruik van 15 - 25\%. Bij het
tweede klik event voor het pauzeren van de audio stijgt het CPU gebruik tot 35\%. 
Bij het derde klik event voor het starten van de video stijgt het CPU gebruik tot 43\% en 
blijft dan schommelen tussen 10 - 35\%. Tijdens het afspelen van de audio en video blijft 
het geheugen gebruik rond de 207MB hangen. Er is dus geen verschil in
geheugen gebruik bij het afspelen van audio of video en het inactief zijn van de applicatie.


\subsubsection{Schaalbaarheid}

\paragraph{Complexiteit}
De gebruikte library react-native-track-player kan nogal complex overkomen. De library 
biedt heel wat mogelijkheden aan die niet altijd nodig zijn. Ook zijn er wel wat stappen 
nodig om de TrackPlayer werkend te krijgen. In sommige situaties kan 
react-native-sound of react-native-sound-player een betere keuze zijn maar in sommige gevallen kan de 
de extra complexiteit van react-native-track-player ook een voordeel zijn. 

De gebruikte library react-native-video is dan weer gemakkelijk te implementeren en te gebruiken.
We moeten enkel de library installeren en krijgen dan direct toegang aan een \textbf{<Video>} tag
die we kunnen gebruiken om video's af te spelen.

\paragraph{Herbruikbaarheid}
Het is mogelijk om alle functionaliteit van bijvoorbeeld de audio en video speler elk in een 
apart component te steken, zodat deze gemakkelijk op andere plaatsen in de applicatie kunnen worden 
gebruikt. Maar dit is niet altijd
nodig. Want in sommige gevallen is het ook mogelijk om de audio en video speler in één component
te steken. Dit is bijvoorbeeld het geval wanneer de audio en video speler dezelfde functionaliteit
hebben. Desondanks is het wel mogelijk om de audio en video speler in aparte componenten te steken
en deze dan te op meerdere plaatsen in de applicatie te gebruiken. Ook is het mogelijk om ze dan 
individueel op te schalen.



\section{Conclusie}
We zien bij het geheugen dat er geen verschil is tussen het afspelen van audio of video 
en het inactief zijn van de applicatie. Desondanks is het geheugengebruik bij cross-platform veel 
hoger dan bij native. er wordt bij native maar 73MB gebruikt in vergelijking met 207MB 
bij cross-platform, een verschil van 183,56\%.
\\\\
Bij het CPU gebruik scoort native opnieuw beter dan cross-platform. Zowel bij het afspelen van audio als video 
is het CPU gebruik bij native lager. Bij het afspelen van audio bij 
native is de beginpiek 17\% en de eindpiek 23\%. Bij cross-platform is de 
beginpiek 42\% en de eindpiek 35\% terwijl het gemiddeld CPU gebruik bij native 0-1\% is en bij
cross-platform 20\%. Bij het afspelen van video bij native is de beginpiek 17\% en de eindpiek 16\%.
Bij cross-platform is de beginpiek 43\% en de eindpiek 28\% terwijl het gemiddeld CPU gebruik bij
native 4\% is en bij cross-platform 22\%. 
\\\\
\begin{tabular}{ |p{3cm}||p{5cm}|p{5cm}| }
    \hline
    \multicolumn{3}{|c|}{Audio} \\ 
    \hline
     & Native (Android) & Cross-platform (React Native) \\
    \hline
     & \multicolumn{2}{|c|}{Geheugen} \\ 
    \hline
    Offset & 2-3MB & 1-2MB \\
    Gemiddeld & 73MB & 207MB \\
    \hline
     & \multicolumn{2}{|c|}{CPU} \\
    \hline
    (Start)Piek & 17\% & 42\% \\
    (Eind)Piek & 23\% & 35\% \\
    Offset & / & 5\% \\
    Gemiddeld & 0-1\% & 20\% \\
    \hline
    \multicolumn{3}{|c|}{Video} \\ 
    \hline
     & Native (Android) & Cross-platform (React Native) \\
    \hline
     & \multicolumn{2}{|c|}{Geheugen} \\ 
    \hline
    Offset & 2-3MB & 1-2MB \\
    Gemiddeld & 73MB & 207MB \\
    \hline
     & \multicolumn{2}{|c|}{CPU} \\
    \hline
    (Start)Piek & 17\% & 43\% \\
    (Eind)Piek & 16\% & 28\% \\
    Offset & 3\% & 12\% \\
    Gemiddeld & 4\% & 22\% \\
    \hline
\end{tabular}
\\\\
De ontwikkeltijd bij
native ligt veel lager lag dan bij cross-platform. Dit komt omdat native gemakkelijker is om te implementeren en te gebruiken.
En omdat er geen extra libraries nodig zijn. Terwijl dit bij cross-platform wel het geval is.
Het kan echter de moeite waard zijn om extra tijd te investeren in cross-platform ontwikkeling 
aangezien de gebruikte bibliotheek extra functionaliteit biedt, zoals bijvoorbeeld de 
mogelijkheid om audio en video in de achtergrond af te spelen. Dit kan helpen bij eventuele 
uitbreidingen in de toekomst.
\\\\
Op basis van de verzamelde gegevens kan geconcludeerd worden dat voor het afspelen van audio en video native de beste keuze is.
Dit komt omdat native beter scoort op vlak van performantie. Ook is native gemakkelijker te implementeren
en te gebruiken. Daarnaast zijn er geen extra libraries nodig voor de implementatie. Let wel op, dit is enkel het geval
wanneer er geen extra functionaliteit nodig is. Indien er extra functionaliteit nodig is, kan
cross-platform een betere keuze zijn. Je zal in dat geval wel met een groot performantie verschil zitten.
















