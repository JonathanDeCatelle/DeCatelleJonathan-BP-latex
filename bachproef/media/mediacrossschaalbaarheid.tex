\paragraph{Complexiteit}
De gebruikte library react-native-track-player kan nogal complex overkomen. De library 
biedt heel wat mogelijkheden aan die niet altijd nodig zijn. Ook zijn er redelijk wat stappen 
nodig om de TrackPlayer werkend te krijgen. In sommige situaties kan 
react-native-sound of react-native-sound-player een betere keuze zijn maar in sommige gevallen kan 
de extra complexiteit van react-native-track-player ook een voordeel zijn. 

De gebruikte library react-native-video is gemakkelijk om te implementeren en te gebruiken.
Na het installeren van de library is er direct toegang tot de \textbf{<Video>} component
die dan gebruikt kan worden om video's af te spelen.

\paragraph{Herbruikbaarheid}
Het is mogelijk om alle functionaliteiten van bijvoorbeeld de audio en video speler elk in een 
aparte component te steken, zodat deze gemakkelijk op andere plaatsen in de applicatie kunnen worden 
gebruikt. Maar dit is niet altijd
nodig. In sommige gevallen is het ook mogelijk om de audio en video speler in één component
te steken. Dit is bijvoorbeeld het geval wanneer de audio en video speler eenzelfde functionaliteit
hebben. Desondanks is het wel mogelijk om de audio en video speler in aparte componenten te steken
en deze dan op meerdere plaatsen in de applicatie te gebruiken. Ook is het mogelijk om ze dan 
individueel op te schalen.
