%%=============================================================================
%% Methodologie
%%=============================================================================

\chapter{Methodologie}
\label{ch:methodologie}

%% TODO: Hoe ben je te werk gegaan? Verdeel je onderzoek in grote fasen, en
%% licht in elke fase toe welke stappen je gevolgd hebt. Verantwoord waarom je
%% op deze manier te werk gegaan bent. Je moet kunnen aantonen dat je de best
%% mogelijke manier toegepast hebt om een antwoord te vinden op de
%% onderzoeksvraag.

Zoals elk onderzoek start ook dit onderzoek met een uitgebreide literatuurstudie van wat mobiele applicaties zijn, 
hoe dat ze worden ontwikkeld en welke functionaliteiten onderzocht gaan worden. Deze literatuurstudie is te vinden in Hoofdstuk \ref{ch:stand-van-zaken}.
\\\\
Vervolgens wordt er in hoofdstuk \ref{ch:ontwikkelomgeving} gekeken hoe dat we de ontwikkelomgeving opstellen om met Kotlin in Android Studio 
te werken en met React native in Visual Studio Code.
\\\\
Na het opstellen van de ontwikkelomgeving wordt er in hoofdstuk \ref{ch:projecten} uitgelegd hoe dat we de blanco projecten aanmaken. 
Deze blanco projecten worden dan doorheen de bachelorproef gebruikt om functionaliteiten uit te werken.
\\\\
Daarna zullen alle functionaliteiten uitgewerkt worden tot een project, hier wordt dan ook de ontwikkeltijd van een functionaliteit 
gedocumenteerd met daarbij eventuele bugs of problemen. Ook wordt hier de performantie van de functionaliteiten gemeten. 
En de eventuele mogelijkheid om op te schalen zal hier ook bekeken worden. Elke functionaliteit wordt opgedeeld in zijn eigen hoofdstuk. 
Op die manier kunnen de resultaten per functionaliteit duidelijk teruggevonden worden.
\begin{itemize}
    \item Hoofdstuk \ref{ch:basisfunctionaliteiten} voor de basisfunctionaliteiten
    \item Hoofdstuk \ref{ch:camera} voor de camera-integratie
    \item Hoofdstuk \ref{ch:sensoren} voor het gebruik van sensoren
    \item Hoofdstuk \ref{ch:notificaties} voor de push-notificaties
    \item Hoofdstuk \ref{ch:audioenvideo} voor de audio- en videospelers
\end{itemize}
Tot slot worden de resultaten en individuele conclusies van alle functionaliteiten uit de vorige hoofdstukken opgelijst en samengevat 
in hoofdstuk \ref{ch:conclusie} om een antwoord te geven op de onderzoeksvragen uit hoofdstuk \ref{ch:inleiding}.

\section{Volgorde onderzoek}
Voor de functionaliteiten te onderzoeken zullen we hetzelfde stappenplan volgen per functionaliteit. 
\begin{enumerate}
    \item Blanco project aanmaken.
    \item Ontwikkeltijd meten.
    \begin{enumerate}
        \item Duurtijd van functionaliteit implementeren.
        \item Duurtijd van eventuele bugs of problemen.
    \end{enumerate}
    \item Performantie van functionaliteit meten.
    \item Mogelijkheid om op te schalen onderzoeken.
    \item resultaten formuleren.
\end{enumerate}
We zullen beginnen door een blanco project aan te maken. Daarna zullen we op dit blanco project de functionaliteit in kwestie implementeren. 
Hierbij gaan we dan de tijd nodig om de functionaliteit te implementeren meten en ook eventuele bugs of problemen. 
Na het implementeren van de functionaliteit zullen we bij beide applicaties de performantie meten. 
Daarna gaan we kijken of dat de functionaliteit opgeschaald kan worden, hoe moeilijk of makkelijk dit kan gebeuren en hoe dat dit moet gebeuren. 
Tot slot zullen we dan per functionaliteit een conclusie formuleren op basis van de verkregen resultaten.

\section{Hoe word er getest}
\subsection{Performantie}
\paragraph{Android Studio}
Om de performantie van de native applicaties te meten maken we gebruik van Android Studio. 
Android Studio bied de tool Android profiler aan die het mogelijk maken om de performantie in kaart te brengen. 
Deze tool kunnen we openen met \textit{View > Tool Windows > Profiler}. %TODO foto invoegen
Deze tool stelt ons in staat om de performantie van applicaties tijdens het uitvoeren op een emulator te analyseren en meten. 
Het geeft ons gedetailleerde inzichten in verschillende prestatieaspecten, zoals CPU-gebruik, geheugengebruik, netwerkactiviteit en energieverbruik. 
Met de Android Profiler kunnen we ook realtime grafieken genereren en de gegevens bekijken. 
\\\\
Doorheen dit onderzoek zullen we gebruik maken van de CPU-profiler en geheugenprofiler om de performantie te meten.

\subparagraph{CPU-profiler}
Hiermee kunnen we de CPU-activiteiten van de applicaties meten. Op die manier kunnen we zien welke threads en methoden de CPU gebruiken en ook hoelang dat ze deze gebruiken.

\subparagraph{Geheugenprofiler}
Hiermee kunnen we het geheugengebruik van de applicaties meten. We kunnen de geheugenallocatie en deallocatie volgen en de geheugenprofielen analyseren. 
Op deze manier kunnen we de impact van de applicatie op het geheugengebruik begrijpen.

\paragraph{React native}
Om de performantie van de cross-platform applicaties te meten maken we gebruik van de Performance Monitor tool. 
Deze is beschikbaar via de React native Developer Tools. Naast de Performance Monitor bieden de React native Developer Tools ook nog andere tools aan zoals: 
Element Inspector, Network Inspector, Console Logging en Redux Debugger. Binnen dit onderzoek gaan we enkel gebruik maken van de Performance Monitor. 
Deze zullen we net zoals bij Android Studio gebruiken om het CPU- en geheugengebruik van cross-platform applicaties te meten en analyseren.

\subsection{Schaalbaarheid}
TODO

\subsection{Ontwikkeltijd}
TODO
