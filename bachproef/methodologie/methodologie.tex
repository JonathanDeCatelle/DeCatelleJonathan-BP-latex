%%=============================================================================
%% Methodologie
%%=============================================================================

\chapter{Methodologie}
\label{ch:methodologie}

In dit hoofdstuk wordt uitgelegd hoe de bachelorproef er zal uitzien, wat 
de volgorde is van het onderzoek en waar er precies op getest wordt.
\\\\
Vervolgens wordt er in hoofdstuk \ref{ch:ontwikkelomgeving} gekeken hoe 
de ontwikkelomgeving wordt opgesteld om met Kotlin in Android Studio 
te werken en met React Native in Visual Studio Code.
\\\\
Na het opstellen van de ontwikkelomgeving wordt er in hoofdstuk 
\ref{ch:projecten} uitgelegd hoe de blanco projecten worden aangemaakt. 
Deze blanco projecten worden dan doorheen de bachelorproef gebruikt om 
functionaliteiten uit te werken.
\\\\
Daarna zullen alle functionaliteiten uitgewerkt worden tot een project. 
Hier wordt dan ook de ontwikkeltijd van een functionaliteit gedocumenteerd 
met daarbij eventuele bugs of problemen. Ook wordt hier de performantie van 
de functionaliteiten gemeten. De eventuele mogelijkheid om op te 
schalen zal hier ook bekeken worden. Elke functionaliteit wordt opgedeeld 
in zijn eigen hoofdstuk. Op die manier kunnen de resultaten per 
functionaliteit duidelijk teruggevonden worden.
\begin{itemize}
    \item Hoofdstuk \ref{ch:basisfunctionaliteiten} voor de basisfunctionaliteiten
    \item Hoofdstuk \ref{ch:audioenvideo} voor de audio- en videospelers
    \item Hoofdstuk \ref{ch:sensoren} voor het gebruik van sensoren
    \item Hoofdstuk \ref{ch:notificaties} voor de push-notificaties
\end{itemize}
Tot slot worden de resultaten en individuele conclusies van alle 
functionaliteiten uit de vorige hoofdstukken opgelijst en samengevat in hoofdstuk 
\ref{ch:conclusie} om een antwoord te geven op de onderzoeksvragen uit 
hoofdstuk \ref{ch:inleiding}.

\section{Volgorde onderzoek}

Om de functionaliteiten te onderzoeken, wordt hetzelfde stappenplan gevolgd per functionaliteit. 
\begin{enumerate}
    \item Aanmaken van blanco project
    \item Ontwikkeltijd meten
    \begin{enumerate}
        \item Duurtijd van de implementatie van een functionaliteit
        \item Duurtijd van eventuele bugs of problemen tijdens het implementeren
    \end{enumerate}
    \item Performantie meten
    \begin{enumerate}
        \item Tijdsduur van uit te voeren code meten
        \item CPU en Geheugen gebruik meten
    \end{enumerate}
    \item Mogelijkheid om op te schalen onderzoeken
    \begin{enumerate}
        \item Mogelijkheid tot abstractie bekijken
        \item Herbruikbaarheid van code
    \end{enumerate}
    \item Conclusie formuleren
\end{enumerate}
Het onderzoek start altijd met het aanmaken van een blanco project. 
Daarna wordt de functionaliteit in dit blanco project geïmplementeerd. 
Hierbij wordt de tijd nodig om de functionaliteit te implementeren 
bijgehouden alsook eventuele bugs of problemen. Na het implementeren van de 
functionaliteit wordt de performantie van beide applicaties gemeten. 
Daarna wordt er gekeken naar de complexiteit van de code en of de functionaliteit 
opgeschaald kan worden. Tot slot wordt er dan per functionaliteit een conclusie 
geformuleerd op basis van de verkregen resultaten.


\section{Hoe wordt er getest}

\subsection{Ontwikkeltijd}
Hoe wordt de ontwikkeltijd van applicaties gemeten doorheen het onderzoek?

\paragraph{Algemene ontwikkeltijd}
Om de algemene ontwikkeltijd van een functionaliteit te meten, wordt bijgehouden 
hoeveel tijd er besteed wordt aan het implementeren. Dit kan dan verder 
opgedeeld worden in de opzet en implementatie van een functionaliteit. 
Daarnaast wordt na het onderzoek de algemene compiletijd besproken en vergeleken. 
Aangezien dit een enorm verschil kan geven op de ontwikkeltijd bij een applicatie. 
Een verschil van vijf minuten voor het zien van veranderingen 
in de applicatie, in vergelijking met twee minuten, kan de algemene 
ontwikkeltijd veel beïnvloeden.

\paragraph{Bugs}
Indien er eventuele bugs voorkomen tijdens het implementeren van een 
functionaliteit, dan zal deze ook beschreven worden: wat is de oorzaak, 
hoe is het opgelost en hoeveel tijd is er hieraan gespendeerd.

\subsection{Performantie}
Hoe wordt de performantie van applicaties gemeten doorheen het onderzoek?

\paragraph{Android Studio}
Het is mogelijk om de performantie van native applicaties te meten met de Android profiler tool binnen Android Studio 
\textit{View > Tool Windows > Profiler}. Deze tool wordt gebruikt om inzicht te verkrijgen 
in verschillende prestatieaspecten, zoals CPU-gebruik, geheugengebruik, netwerkactiviteit en 
energieverbruik. De Android Profiler genereert realtime grafieken om de gegevens te bekijken. 

\paragraph{React native}
Om de performantie van de React native applicaties te meten kan de Performance 
Monitor tool gebruikt worden. Deze is beschikbaar via de React native Developer 
Tools en is geoptimaliseerd om de werking en performantie van React native 
componenten te meten. Naast de Performance Monitor bieden de React native Developer 
Tools ook nog andere tools aan zoals: Element Inspector, Network Inspector, Console 
Logging en Redux Debugger. De Performance Monitor tool is specifiek ontworpen 
voor het meten en analyseren van de performantie

\subparagraph{Probleem performantietools}
Normaal gezien worden React native applicaties getest met de Performance Monitor tool.
Maar ondanks dat deze inzicht geeft in de performantie van de applicatie, 
zijn er een aantal verschillen met de Android profiler. De Performance Monitor tool 
van React native is namelijk zodanig ontworpen om enkel de performantie van de 
react applicaties te meten en analyseren. Op die manier kunnen eventuele 
\gls{bottlenecks} ontdekt worden binnen de applicatie om zo de performantie te verbeteren.
\\\\
Het probleem bij dit onderzoek is dat de performantie van beide applicaties met 
elkaar vergeleken moet worden. En met de Performance Monitor tool van React 
native is dit niet mogelijk. Daarom dient er gezocht te worden naar een externe 
een alternatieve manier om de performantie op een eerlijke manier te vergelijken.

\subparagraph{Firebase Performance Monitoring}
Één manier om de performantie te meten is de Firebase Performance Monitoring tool. 
Deze is onwtikkeld door Google en kan de duur van een stuk code van beide applicaties 
onafhankelijk meten. Dankzij deze tool kunnen we bijvoorbeeld opstarttijd, HTTP requests, 
schermlaadtijd, enz... meten. De implementatie van deze library voor beide ontwikkelmethodes wordt besproken 
in hoofdstuk \ref{ch:projecten}.

\subparagraph{Android profiler}
Ondanks dat de Firebase Performance Monitoring gebruikt kan worden om de tijdsduur 
van een stuk code te meten, is dit niet genoeg om een volwaardige vergelijking te 
doen. Daarom wordt de Android profiler ook voor de React native applicaties 
gebruikt. Dit is mogelijk als de emulator binnen Android Studio eerst wordt opgestart 
en daarna pas de React native commandos om een applicatie te starten worden uitgevoerd 
\ref{par:emulatorgebruiken}. Hierdoor wordt de applicatie in de emulator 
binnen Android Studio opgestart en kan de Android profiler gebruikt worden. 
\\\\
Door het gebruik van zowel de Android Profiler als Firebase Performance Monitoring 
kunnen de prestaties van applicaties op een eerlijke manier worden vergeleken

\subsection{Schaalbaarheid}
Hoe wordt de schaalbaarheid van applicaties gemeten doorheen het onderzoek?

\paragraph{Complexiteit}
Eerst wordt de complexiteit van de code geanalyseerd om te kijken hoe goed deze 
gestructureerd en georganiseerd is. Daarnaast wordt er ook gekeken of de 
functionaliteit in kleine herbruikbare componenten opgedeeld kan worden aangezien een 
goed gestructureerde codebase, opgedeeld in kleine componenten, gemakkelijker 
is om aan te passen en uit te breiden.

\paragraph{Herbruikbaarheid}
Tot slot wordt ook gekeken in hoeverre de code van de functionaliteiten hergebruikt kan 
worden in andere delen van de applicatie. Want als componenten gemakkelijk hergebruikt 
kunnen worden, vergroot dit de schaalbaarheid van de applicatie omdat 
toekomstige aanpassingen of uitbreidingen gemakkelijker geïmplementeerd kunnen worden.


