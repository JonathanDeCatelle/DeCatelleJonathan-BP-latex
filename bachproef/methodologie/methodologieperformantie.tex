\subsection{Performantie}
Hoe wordt de performantie van applicaties gemeten doorheen het onderzoek?

\paragraph{Android Studio}
Het is mogelijk om de performantie van native applicaties te meten met de Android profiler tool binnen Android Studio 
\textit{View > Tool Windows > Profiler}. Deze tool wordt gebruikt om inzicht te verkrijgen 
in verschillende prestatieaspecten, zoals CPU-gebruik, geheugengebruik, netwerkactiviteit en 
energieverbruik. De Android Profiler genereert realtime grafieken om de gegevens te bekijken. 

\paragraph{React native}
Om de performantie van de React native applicaties te meten kan de Performance 
Monitor tool gebruikt worden. Deze is beschikbaar via de React native Developer 
Tools en is geoptimaliseerd om de werking en performantie van React native 
componenten te meten. Naast de Performance Monitor bieden de React native Developer 
Tools ook nog andere tools aan zoals: Element Inspector, Network Inspector, Console 
Logging en Redux Debugger. De Performance Monitor tool is specifiek ontworpen 
voor het meten en analyseren van de performantie

\subparagraph{Probleem performantietools}
Normaal gezien worden React native applicaties getest met de Performance Monitor tool.
Maar ondanks dat deze inzicht geeft in de performantie van de applicatie, 
zijn er een aantal verschillen met de Android profiler. De Performance Monitor tool 
van React native is namelijk zodanig ontworpen om enkel de performantie van de 
react applicaties te meten en analyseren. Op die manier kunnen eventuele 
\gls{bottlenecks} ontdekt worden binnen de applicatie om zo de performantie te verbeteren.
\\\\
Het probleem bij dit onderzoek is dat de performantie van beide applicaties met 
elkaar vergeleken moet worden. En met de Performance Monitor tool van React 
native is dit niet mogelijk. Daarom dient er gezocht te worden naar een externe 
een alternatieve manier om de performantie op een eerlijke manier te vergelijken.

\subparagraph{Firebase Performance Monitoring}
Één manier om de performantie te meten is de Firebase Performance Monitoring tool. 
Deze is onwtikkeld door Google en kan de duur van een stuk code van beide applicaties 
onafhankelijk meten. Dankzij deze tool kunnen we bijvoorbeeld opstarttijd, HTTP requests, 
schermlaadtijd, enz... meten. De implementatie van deze library voor beide ontwikkelmethodes wordt besproken 
in hoofdstuk \ref{ch:projecten}.

\subparagraph{Android profiler}
Ondanks dat de Firebase Performance Monitoring gebruikt kan worden om de tijdsduur 
van een stuk code te meten, is dit niet genoeg om een volwaardige vergelijking te 
doen. Daarom wordt de Android profiler ook voor de React native applicaties 
gebruikt. Dit is mogelijk als de emulator binnen Android Studio eerst wordt opgestart 
en daarna pas de React native commandos om een applicatie te starten worden uitgevoerd 
\ref{par:emulatorgebruiken}. Hierdoor wordt de applicatie in de emulator 
binnen Android Studio opgestart en kan de Android profiler gebruikt worden. 
\\\\
Door het gebruik van zowel de Android Profiler als Firebase Performance Monitoring 
kunnen de prestaties van applicaties op een eerlijke manier worden vergeleken
