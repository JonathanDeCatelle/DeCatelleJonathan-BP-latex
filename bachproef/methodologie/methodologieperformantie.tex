\subsection{Performantie}
Hoe gaan we de performantie van applicaties doorheen het onderzoek meten?

\paragraph{Android Studio}
De performantie van native applicaties kan worden gemeten met de Android profiler tool binnen Android Studio 
\textit{View > Tool Windows > Profiler}. Deze tool stelt ons in staat om de performantie van applicaties 
tijdens het uitvoeren te analyseren en meten. Het geeft ons gedetailleerde inzichten 
in verschillende prestatieaspecten, zoals CPU-gebruik, geheugengebruik, netwerkactiviteit en 
energieverbruik. Met de Android Profiler kunnen we ook realtime grafieken genereren en de gegevens bekijken. 

\paragraph{React native}
Om de performantie van de React native applicaties te meten kunnen we gebruik maken van de Performance Monitor tool. 
Deze is beschikbaar via de React native Developer Tools en is geoptimaliseerd om de werking en performantie van React native 
componenten te meten. Naast de Performance Monitor bieden de React native Developer 
Tools ook nog andere tools aan zoals: Element Inspector, Network Inspector, Console Logging en Redux Debugger. De 
Performance Monitor tool is specifiek ontworpen voor het meten en analyseren van de performantie

\subparagraph{Probleem performantietools}
Normaal gezien indien je de React native applicaties zou testen, zou je gebruik maken van de Performance Monitor tool.
Maar ondanks dat deze inzicht geeft in de performantie van de applicatie, 
zijn er een aantal verschillen met de Android profiler. De Performance Monitor tool van React native is namelijk zodanig 
ontworpen om enkel de performantie van de react applicaties te meten en analyseren.
Op die manier kunnen eventuele \gls{bottlenecks} ontdekt worden binnen de applicatie om zo de performantie te verbeteren.
\\\\
Het probleem voor dit onderzoek is dat de performantie van beide applicaties met elkaar vergeleken moet worden. 
En met de Performance Monitor tool van React native is dit niet mogelijk. Daarom moeten we hiervoor kijken naar 
een externe library die deze taak op zich kan nemen of een andere manier om toch de performantie op een eerlijke manier 
te vergelijken. 

\subparagraph{Firebase Performance Monitoring}
Één manier om de performantie te meten is de Firebase Performance Monitoring tool \url{https://firebase.google.com/docs/perf-mon/}. 
Deze is onwtikkeld door Google en stelt ons in staat om de duur van een stuk code van beide applicaties 
onafhankelijk te meten en vergelijken met elkaar. Dankzij deze tool kunnen we bijvoorbeeld opstarttijd, HTTP requests, 
schermlaadtijd, enz\dots meten. De implementatie van deze library voor beide ontwikkelmethodes wordt besproken 
in hoofdstuk \ref{ch:projecten}.

\subparagraph{Android profiler}
Ondanks dat we de Firebase Performance Monitoring kunnen gebruiken om de tijdsduur van een stuk code te meten, is dit 
niet genoeg om een volwaardige vergelijking te doen. Daarom gaan we ook voor de React native applicaties de Android 
profiler gebruiken. Dit is mogelijk als we de emulator binnen Android Studio eerst opstarten en daarna pas de React native 
commandos om een applicatie te starten uitvoeren \ref{par:emulatorgebruiken}. Hierdoor wordt de applicatie in de emulator 
binnen Android Studio opgestart en kunnen we terug gebruik maken van de Android profiler. 
\\\\
Door een combinatie van Android profiler en Firebase Performance Monitoring te gebruiken kunnen we op een eerlijke manier 
de performantie tussen beide applicaties vergelijken met elkaar.