Voor de functionaliteiten te onderzoeken zullen we hetzelfde stappenplan volgen per functionaliteit. 
\begin{enumerate}
    \item Blanco project aanmaken.
    \item Ontwikkeltijd meten.
    \begin{enumerate}
        \item Duurtijd van functionaliteit implementeren.
        \item Duurtijd van eventuele bugs of problemen.
    \end{enumerate}
    \item Performantie van functionaliteit meten.
    \begin{enumerate}
        \item Duur van uit te voeren code meten.
        \item CPU en Geheugen gebruik meten.
    \end{enumerate}
    \item Mogelijkheid om op te schalen onderzoeken.
    \begin{enumerate}
        \item mogelijkheid tot abstractie bekijken.
        \item Herbruikbaarheid van code testen.
    \end{enumerate}
    \item conclusie formuleren.
\end{enumerate}
We zullen beginnen door een blanco project aan te maken. Daarna zullen we op dit blanco project de 
functionaliteit in kwestie implementeren. 
Hierbij gaan we dan de tijd nodig om de functionaliteit te implementeren meten en ook eventuele bugs of problemen. 
Na het implementeren van de functionaliteit zullen we bij beide applicaties de performantie meten. 
Daarna gaan we kijken of dat de functionaliteit opgeschaald kan worden, hoe moeilijk of makkelijk dit kan gebeuren 
en hoe dat dit moet gebeuren. 
Tot slot zullen we dan per functionaliteit een conclusie formuleren op basis van de verkregen resultaten.