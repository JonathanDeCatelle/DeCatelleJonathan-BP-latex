Om de functionaliteiten te onderzoeken, wordt hetzelfde stappenplan gevolgd per functionaliteit. 
\begin{enumerate}
    \item Aanmaken van blanco project
    \item Ontwikkeltijd meten
    \begin{enumerate}
        \item Duurtijd van de implementatie van een functionaliteit
        \item Duurtijd van eventuele bugs of problemen tijdens het implementeren
    \end{enumerate}
    \item Performantie meten
    \begin{enumerate}
        \item Tijdsduur van uit te voeren code meten
        \item CPU en Geheugen gebruik meten
    \end{enumerate}
    \item Mogelijkheid om op te schalen onderzoeken
    \begin{enumerate}
        \item Mogelijkheid tot abstractie bekijken
        \item Herbruikbaarheid van code
    \end{enumerate}
    \item Conclusie formuleren
\end{enumerate}
Het onderzoek start altijd met het aanmaken van een blanco project. 
Daarna wordt de functionaliteit in dit blanco project geïmplementeerd. 
Hierbij wordt de tijd nodig om de functionaliteit te implementeren 
bijgehouden alsook eventuele bugs of problemen. Na het implementeren van de 
functionaliteit wordt de performantie van beide applicaties gemeten. 
Daarna wordt er gekeken naar de complexiteit van de code en of de functionaliteit 
opgeschaald kan worden. Tot slot wordt er dan per functionaliteit een conclusie 
geformuleerd op basis van de verkregen resultaten.
