\paragraph{Tijdsduur}
\begin{figure}[H]
    \centering
    \includegraphics[height=0.085\textheight]{notificatiesDuratieNative.png}
    \caption{Overzicht tijdsduur aanmaken notificaties bij Android.}
\end{figure}
Tijdens het meten van de duur voor het aanmaken van een notificatie, is er 
10 keer een notificatie aangemaakt. Na 10 keer een notificatie aan te maken, is er  
een gemiddelde duur van 12ms en met een minimum en 
maximum van 4ms en 88ms.

\paragraph{CPU \& geheugen}
\begin{figure}[H]
    \centering
    \includegraphics[height=0.25\textheight]{notificatiesPerformantieNative.png}
    \caption{Overzicht CPU en geheugen gebruik tijdens aanmaken notificaties bij Android.}
\end{figure}
Op de grafiek is te zien dat het CPU gebruik van de applicatie rond de 4\% ligt wanneer deze inactief is. 
Het is ook duidelijk zichtbaar wanneer een notificatie wordt aangemaakt. De piek van het CPU gebruik lag 
gemiddeld op 21\% en schommelde tussen de 19\% en 27\%. Het geheugen blijft in tegenstelling tot de CPU 
rond de 81MB hangen, wanneer de applicatie inactief en actief is, met verschillen van maximum 2-3MB. Er is geen 
merkbaar verschil in het geheugen wanneer een notificatie wordt aangemaakt.
  