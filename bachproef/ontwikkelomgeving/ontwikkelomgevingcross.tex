Om cross-platform te ontwikkelen door gebruik te maken van React Native is er wat meer werk om 
de omgeving op te stellen. Eerst zullen we Visual Studio Code downloaden. 
Dit kan vanop de website \url{https://code.visualstudio.com/download} 

Na het installeren van Visual Studio Code kunnen we de React Native ontwikkelomgeving opstellen. 
Dit kan door de handleiding te volgen vanop de website \url{https://reactnative.dev/docs/environment-setup}.

\paragraph{1. React Native CLI Quickstart}
Voor de React Native ontwikkelomgeving kunnen we ook gebruik maken van Expo. 
Dit is een simpele manier om snel applicaties visueel werkend te krijgen. 
Maar voor dit onderzoek zullen we de applicaties op dezelfde emulator runnen waarop de native 
ontwikkelde applicaties runnen. Hiervoor volgen we de React Native CLI Quickstart handleiding. 

\paragraph{2. Besturingssysteem en doelsysteem}
Afhankelijk van het gebruikte besturingssysteem is er een andere handleiding. 
Doorheen dit onderzoek zullen we werken op een Windows laptop en zullen we focussen op Android applicaties.

\paragraph{3. Node \& JDK}
Als eerste moeten we \Gls{Node} en \acrshort{jdk} installeren. Dit kunnen we gemakkelijk 
doen met \Gls{Chocolatey}. Chocolatey indien niet geïnstalleerd kan worden gedownload 
vanop de website \url{https://chocolatey.org/install}. Eerst openen we een 
opdrachtprompt (powershell) als administrator waaraan we volgende commando meegeven.
\begin{minted}{bash}
choco install -y nodejs-lts microsoft-openjdk11
\end{minted}
Voor Node gebruiken we de laatste versie. De minimale versie waarmee React Native kan werken is 14. 
Voor de JDK gebruiken we versie 11 aangezien hogere versies voor problemen kunnen zorgen.

\paragraph{4. Android ontwikkelomgeving}
Net zoals bij native ontwikkeling hebben we voor React Native applicaties ook Android Studio nodig. 
Dit is omdat we om de applicatie te runnen gebruik maken van de emulator binnenin Android Studio. 
Android Studio zelf wordt niet direct gebruikt maar er wordt enkel gebruik gemaakt van de 
emulator die Android Studio aanbied.
\\\\
Bij het opstellen van de React Native omgeving zijn er een paar instellingen waar we rekening 
mee moeten houden. Maar hiermee is al rekening gehouden bij het installeren van Android Studio 
in paragraaf \ref{par:sdk}.

\paragraph{5. ANDROID\_HOME omgevingsvariabelen}
De React Native tools hebben enkele omgevingsvariabelen nodig om native applicaties te bouwen. 
Om deze omgevingsvariabelen in te stellen openen we eerst het configuratiescherm van Windows. 
Daarna gaan we naar 
\textit{Gebruikersaccounts > Gebruikersaccounts > Mijn omgevingsvariabelen veranderen > Nieuw\dots}. 
In het nieuwe verkregen scherm vullen we dan de naam van de omgevingsvariabel \textbf{ANDROID\_HOME} 
en de path naar de Android SDK in. De standaard path naar de Android SDK is 
\textbf{\%LOCALAPPDATA\%\backslash Android\backslash Sdk}. 
De locatie van de Android SDK is ook te vinden via Android Studio via 
\textit{File > Settings > Appearance \& Behavior > System Settings > Android SDK}.
\\\\
Om te controleren dat de omgevingsvariabel correct is toegevoegd openen we een nieuwe 
opdrachtprompt (powershell) en geven we volgend commando in.
\begin{minted}{bash}
Get-ChildItem -Path Env:\
\end{minted}
In de teruggegeven lijst controleren we dan of dat ANDROID\_HOME effectief is toegevoegd.
\\\\
Tot slot moeten we de \textbf{Path} omgevingsvariabel aanpassen. 
Opnieuw navigeren we in het configuratiescherm naar het overzicht met alle omgevingsvariabelen 
\textit{Gebruikersaccounts > Gebruikersaccounts > Mijn omgevingsvariabelen veranderen}. 
Hierin selecteren we de Path omgevingsvariabel en daarna drukken we op \textit{Bewerken...}. 
Bij het nieuw verkregen scherm drukken we op \textit{Nieuw} en voegen we de path naar platform-tools 
toe aan de lijst. Het path van de platform-tool is standaard 
\textbf{\%LOCALAPPDATA\%\backslash Android\backslash Sdk\backslash platform-tools}.

\paragraph{6. React Native Command Lin Interface (CLI)}
React Native heeft een ingebouwde command line interface (CLI) om te werken met React naNativetive. 
Deze kan gebruikt worden dankzij Node.js dat we daarnet installeerden.
\begin{minted}{bash}
npx react native [commando]
\end{minted}

\paragraph{7. Emulator gebruiken} \label{par:emulatorgebruiken}
Om nu effectief applicaties te laten runnen maken we gebruik van de emulator die Android Studio aanbied. 
Aangezien deze al opgesteld is bij het opzetten van Android Studio \ref{se:native} 
moeten we dit niet meer doen voor React Native. Om de applicatie starten hebben we twee terminals nodig 
binnen Visual Studio code, deze kunnen we openen via \textit{Terminal > New Terminal}. 
In de eerste terminal starten we \Gls{Metro} met het volgende commando.
\begin{minted}{bash}
npx react-native start
\end{minted}
In de tweede terminal geven we het volgende commando dat de applicatie zal opbouwen en 
deployen naar de emulator.
\begin{minted}{bash}
npx react-native run-android
\end{minted}