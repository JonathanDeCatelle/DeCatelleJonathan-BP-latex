Om cross-platform te ontwikkelen door gebruik te maken van React Native is er wat meer werk om 
de omgeving op te stellen. Eerst moet Visual Studio Code worden gedownload. 
Na het installeren van Visual Studio Code kan de React Native ontwikkelomgeving worden opgesteld. 

\paragraph{1. React Native CLI Quickstart}
Voor de React Native ontwikkelomgeving kan ook gebruik worden gemaakt van Expo. 
Dit is een simpele manier om snel applicaties visueel werkend te krijgen. 
Maar voor dit onderzoek moeten de applicaties op dezelfde emulator runnen waarop we de native 
ontwikkelde applicaties runnen. Daarom wordt Expo niet gebruik.

\paragraph{2. Besturingssysteem en doelsysteem}
Afhankelijk van het gebruikte besturingssysteem is er een andere handleiding. 
Doorheen dit onderzoek zullen we werken op een Windows laptop en zullen we focussen op Android applicaties.

\paragraph{3. Node \& JDK}
Om te beginnen moeten \Gls{Node} en \acrshort{jdk} geïnstalleerd worden. Dit kan gemakkelijk 
worden gedaan met \Gls{Chocolatey}. Eerst wordt een 
opdrachtprompt (powershell) als administrator geopend waaraan volgende commando wordt meegegeven.
\begin{minted}{bash}
choco install -y nodejs-lts microsoft-openjdk11
\end{minted}
Voor Node wordt de laatste versie gebruikt. De minimale versie waarmee React Native kan werken is 14. 
Voor de JDK wordt versie 11 gebruikt aangezien hogere versies voor problemen kunnen zorgen.

\paragraph{4. Android ontwikkelomgeving}
Net zoals bij native ontwikkeling hebben de React Native applicaties ook Android Studio nodig. 
Zo kan de emulator in Android Studio gebruikt worden. 
Android Studio zelf wordt niet direct gebruikt maar er wordt wel gebruik gemaakt van de 
emulator die Android Studio aanbiedt.
\\\\
Bij het opstellen van de React Native omgeving zijn er een paar instellingen waar er rekening 
mee moeten worden gehouden. Hiermee is al rekening gehouden bij het installeren van Android Studio 
in paragraaf \ref{par:sdk}.

\paragraph{5. ANDROID\_HOME omgevingsvariabelen}
De React Native tools hebben enkele omgevingsvariabelen nodig om native applicaties te bouwen. 
Om deze omgevingsvariabelen in te stellen wordt het configuratiescherm van Windows geopend. 
Daarna wordt er genavigeerd naar
\textit{Gebruikersaccounts > Gebruikersaccounts > Mijn omgevingsvariabelen veranderen > Nieuw\dots}. 
In het nieuwe verkregen scherm wordt de naam van de omgevingsvariabel \textbf{ANDROID\_HOME} 
en de path naar de Android SDK ingevuld. De standaard path naar de Android SDK is 
\textbf{\%LOCALAPPDATA\%\backslash Android\backslash Sdk}. 
De locatie van de Android SDK is ook te vinden via Android Studio via 
\textit{File > Settings > Appearance \& Behavior > System Settings > Android SDK}.
\\\\
Om te controleren of dat de omgevingsvariabel correct is toegevoegd wordt een nieuwe 
opdrachtprompt (powershell) geopend en wordt volgend commando ingevoerd.
\begin{minted}{bash}
Get-ChildItem -Path Env:\
\end{minted}
In de teruggegeven lijst kan dan gecontroleerd worden of dat \textbf{ANDROID\_HOME} effectief werd toegevoegd.
\\\\
Tot slot moet de \textbf{Path} omgevingsvariabel aangepast worden. 
Hiervoor wordt opnieuw genavigeerd naar het overzicht met alle omgevingsvariabelen in het configuratiescherm 
\textit{Gebruikersaccounts > Gebruikersaccounts > Mijn omgevingsvariabelen veranderen}. 
Daarna wordt wordt de Path omgevingsvariabel geselecteerd en wordt deze bewerkt met de \textit{Bewerken...} knop. 
Bij het nieuw verkregen scherm wordt op \textit{Nieuw} gedrukt en wordt de path naar platform-tools 
toegevoegd aan de lijst. Het path van de platform-tool is standaard 
\textbf{\%LOCALAPPDATA\%\backslash Android\backslash Sdk\backslash platform-tools}.

\paragraph{6. React Native Command Lin Interface (CLI)}
React Native heeft een ingebouwde command line interface (CLI) om te werken met React naNativetive. 
Deze kan gebruikt worden dankzij Node.js dat daarnet werd installeerden.
\begin{minted}{bash}
npx react native [commando]
\end{minted}

\paragraph{7. Emulator gebruiken} \label{par:emulatorgebruiken}
Om nu effectief applicaties te laten runnen, wordt de emulator die Android Studio aanbiedt gebruikt. 
Aangezien deze al opgesteld is bij het opzetten van Android Studio \ref{se:native}, 
moeten deze niet meer voor React Native worden opgezet. Om de applicatie te starten zijn er twee terminals nodig 
binnen Visual Studio code, deze kunnen we openen via \textit{Terminal > New Terminal}. 
In de eerste terminal wordt \Gls{Metro} gestart met volgende commando:
\begin{minted}{bash}
npx react-native start
\end{minted}
In de tweede terminal wordt de applicatie opgebouwd en geïnstalleerd op de emulator met volgende commando:
\begin{minted}{bash}
npx react-native run-android
\end{minted}
