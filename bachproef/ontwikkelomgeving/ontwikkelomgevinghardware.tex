Doorheen het onderzoek zullen alle projecten starten vanuit een blanco project en op dezelfde emulator 
getest worden. Deze emulator zal altijd runnen op hetzelfde apparaat. Dankzij volgend commando 
kunnen we de informatie verkrijgen van over een apparaat.
\begin{minted}{bash}
npx react-native info
\end{minted}
Deze geeft bij het gebruikte apparaat voor dit onderzoek volgende output.
\begin{minted}{bash}
System:
    OS: Windows 10 10.0.19044
    CPU: (12) x64 AMD Ryzen 5 5600H with Radeon Graphics
    Memory: 2.90 GB / 15.86 GB
Binaries:
    Node: 18.16.0 - C:\Program Files\nodejs\node.EXE
    Yarn: Not Found
    npm: 9.5.1 - C:\Program Files\nodejs\npm.CMD
    Watchman: Not Found
SDKs:
    Android SDK: Not Found
    Windows SDK: Not Found
IDEs:
    Android Studio: AI-222.4459.24.2221.9971841
    Visual Studio: Not Found
Languages:
    Java: 11.0.18
npmPackages:
    @react-native-community/cli: Not Found
    react: 18.2.0 => 18.2.0 
    react-native: 0.71.7 => 0.71.7 
    react-native-windows: Not Found
    npmGlobalPackages:
    *react-native*: Not Found
\end{minted}
We werken dus met Windows 10 en maken gebruik van een apparaat met een AMD Ryzen 5 5600H processor 
met 12 cores, geïntegreerde Radeon Graphics en 16GB RAM geheugen, wat sterk en snel 
genoeg is om het onderzoek uit te voeren.