\subparagraph{1. Configuratie bestanden toevoegen}
In het Firebase project wordt een nieuwe Android applicatie toegevoegd. Om deze aan te maken, moet de package naam 
worden meegegeven die te vinden is in het \textbf{build.gradle(module)} bestand als \textbf{applicationId}.
Optioneel kan de naam van de applicatie mee worden gegeven.
\\\\
Nadat de Android applicatie is aangemaakt, moet het \textbf{google-services.json} bestand worden gedownload. 
Deze wordt dan geplaatst in de \textit{app} folder.

\subparagraph{2. Firebase plugins configureren}
Om Firebase het configuratiebestand nu te laten gebruiken, moet de google-services plugin worden toegevoegd aan 
de dependancies in het \textbf{build.gradle(project)} bestand boven de plugins. 
\begin{minted}{java}
buildscript {
    repositories {
        google()
        mavenCentral()
    }

    dependencies {
        // andere dependancies
        classpath "com.google.gms:google-services:4.3.15"
    }
}
\end{minted}
Daarna moet de plugin aan het \textbf{build.gradle(module)} bestand worden toegevoegd.
\begin{minted}{java}
plugins {
    // andere plugins 
    id "com.google.gms.google-services"
}
\end{minted}
Tot slot moeten de Firebase SDKs worden toegevoegd aan het \textbf{build.gradle(module)} bestand.
\begin{minted}{java}
dependencies {
    // andere dependancies
    implementation platform("com.google.firebase:firebase-bom:32.0.0")
}
\end{minted}
Firebase is nu volledig aan ons project toegevoegd.