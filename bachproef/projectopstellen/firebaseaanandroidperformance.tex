\subparagraph{1. Performance Monitoring SDK aan applicatie toevoegen}
Als eerst voegen we de Performance Monitor SDK toe aan ons project in het \textbf{build.gradle(module)} bestand.
\begin{minted}{java}
dependencies {
    // ... andere dependancies
    implementation "com.google.firebase:firebase-perf-ktx"
}
\end{minted}

\subparagraph{2. Performance Monitoring Gradle plugin toevoegen}
Na het toevoegen van de SDK moeten we de Perormance Monitoring Gradle plugin toevoegen. 
Dit doen we door de dependancies hiervan toe te voegen aan het \textbf{build.gradle(project)} bestand.
\begin{minted}{java}
dependencies {
    // ... andere dependancies
    classpath "com.google.firebase:perf-plugin:1.4.2"
}
\end{minted}
Tot slot voegen we de plugin toe. Dit doen we door deze in het \textbf{build.gradle(module)} bestand toe te voegen.
\begin{minted}{java}
plugins {
    // ... andere plugins
    id "com.google.firebase.firebase-perf"
}
\end{minted}
Nu is het blanco project klaar om vanuit dit project alle functionaliteiten te 
implementeren en onderzoeken, buiten de basisfunctionaliteiten. Hiervoor zullen we vanuit een ander project starten, 
maar zullen we de Perormance Monitoring tool op dezelfde manier implementeren.

\subparagraph{3. Performance meten}
Nu kunnen we een trace maken om de performantie te meten. Doorheen de applicaties zal dit er als 
volgt uitzien.
\begin{minted}{java}
val trace = FirebasePerformance.getInstance().newTrace("naam_trace")

trace.start()
// uit te voeren code
trace.stop()
\end{minted}
De trace wordt gestart voor de code die we willen meten en gestopt na de code die we willen meten.
De trace wordt dan automatisch naar de Firebase console gestuurd. Hier kunnen we de trace terugvinden
en de performantie analyseren.