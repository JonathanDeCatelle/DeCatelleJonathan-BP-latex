\subparagraph{1. Performance Monitoring SDK aan applicatie toevoegen}
Eerst moet de Performance Monitor SDK aan het project worden toegevoegd in het \textbf{build.gradle(module)} bestand.
\begin{minted}{java}
dependencies {
    // andere dependancies
    implementation "com.google.firebase:firebase-perf-ktx"
}
\end{minted}

\subparagraph{2. Performance Monitoring Gradle plugin toevoegen}
Na het toevoegen van de SDK moet de Perormance Monitoring Gradle plugin worden toegevoegd 
aan het \textbf{build.gradle(project)} bestand.
\begin{minted}{java}
dependencies {
    // andere dependancies
    classpath "com.google.firebase:perf-plugin:1.4.2"
}
\end{minted}
Tot slot moet de plugin worden toegevoegd aan het \textbf{build.gradle(module)} bestand.
\begin{minted}{java}
plugins {
    // andere plugins
    id "com.google.firebase.firebase-perf"
}
\end{minted}
Nu is het blanco project klaar om vanuit dit project alle functionaliteiten te 
implementeren en onderzoeken, buiten de basisfunctionaliteiten. Hiervoor wordt er vanuit een ander project gestart, 
maar zal de Performance Monitoring tool op dezelfde manier worden geïmplementeerd.

\subparagraph{3. Performance meten}
Om de performantie te meten doorheen de applicaties zal dit er als 
volgt uitzien:
\begin{minted}{java}
val trace = FirebasePerformance.getInstance().newTrace("naam_trace")

trace.start()
// uit te voeren code
trace.stop()
\end{minted}
De trace wordt automatisch naar de Firebase console gestuurd, van waarop de 
resultaten kunnen worden bekeken.