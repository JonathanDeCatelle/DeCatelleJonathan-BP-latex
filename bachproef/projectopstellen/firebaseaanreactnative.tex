\subparagraph{1. Dependancy installeren}
Eerst moet de React Native Firebase app module aan de root van het React Native project worden toegevoegd. 
Dit wordt gedaan met volgend commando:
\begin{minted}{bash}
npm install --save @react-native-firebase/app
\end{minted}

\subparagraph{2. Configuratie bestanden toevoegen}
Nadat de dependancy is toegevoegd, moet er een nieuw Firebase project worden aangemaakt. 
Om deze aan te maken, moet de package naam worden meegegeven die te vinden is in het 
\textit{android/app/build.gradle} bestand als 
\textbf{applicationId}. Optioneel kan de naam van de applicatie worden meegegeven.
\\\\
Nadat de Android applicatie is aangemaakt, moet het \textbf{google-services.json} bestand worden gedownload. 
Dit wordt dan geplaatst in de \textit{android/app} folder.

\subparagraph{3. Firebase configureren}
Om Firebase het configuratiebestand nu te laten gebruiken, moet de google-services plugin worden toegevoegd aan 
de dependancies binnen het \textit{android/build.gradle} bestand. 
\begin{minted}{java}
buildscript {
    dependencies {
        // andere dependencies
        classpath "com.google.gms:google-services:4.3.15"
    }
}
\end{minted}
Daarna moet de plugin aan het \textit{android/app/build.gradle} bestand worden toegevoegd.
\begin{minted}{java}
apply plugin: "com.google.gms.google-services"
\end{minted}
Tot slot wordt deze ook aan de dependancies toegevoegd in hetzelfde \textit{android/app/build.gradle} bestand.
\begin{minted}{java}
dependencies {
    // andere dependancies
    implementation platform("com.google.firebase:firebase-bom:32.0.0")
}
\end{minted}
De React Native Firebase app module is nu volledig geïmplementeerd.