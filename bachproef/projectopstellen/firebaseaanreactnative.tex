\subparagraph{1. Dependancy installeren}
Daarna moeten we de React Native Firebase "app" module aan ons React native project toevoegen. 
Dit kunnen we doen met volgend commando.
\begin{minted}{bash}
npm install --save @react-native-firebase/app
\end{minted}

\subparagraph{2. Configuratie bestanden toevoegen}
Na dat de dependancy is toegevoegd maken we in het Firebase project opnieuw een Android applicatie aan. 
Om deze aan te maken geven we de package naam dat te vinden is in \textit{android/app/build.gradle} als 
applicationId en optioneel de naam van de applicatie mee.
\\\\
Na dat de Android applicatie is aangemaakt, moeten we het \textbf{google-services.json} bestand downloaden. 
Deze plaatsen we dan in de \textit{android/app} folder.

\subparagraph{3. Firebase configureren}
Om Firebase het configuratiebestand nu te laten gebruiken moeten we de google-services plugin toevoegen aan 
de dependancies binnen het \textit{/android/build.gradle} bestand. 
\begin{minted}{java}
buildscript {
    dependencies {
        // andere dependencies
        classpath "com.google.gms:google-services:4.3.15"
    }
}
\end{minted}
Tot slot moeten we de plugin uitvoeren door deze aan het \textit{/android/app/build.gradle} bestand toe te voegen.
\begin{minted}{java}
apply plugin: "com.google.gms.google-services"
\end{minted}
En voegen we deze ook toe aan de dependancies in hetzelfde \textit{/android/app/build.gradle} bestand.
\begin{minted}{java}
dependencies {
    // andere dependancies
    implementation platform("com.google.firebase:firebase-bom:32.0.0")
}
\end{minted}
De React Native Firebase app module is nu volledig geïmplementeerd.