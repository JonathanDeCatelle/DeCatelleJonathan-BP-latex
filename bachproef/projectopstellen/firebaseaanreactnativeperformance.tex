\subparagraph{1. Dependancy installeren}
Eerst moet de Performance Monitoring tool aan de root van ons project worden toegevoegd. 
Dit wordt gedaan met volgend commando:
\begin{minted}{bash}
npm install --save @react-native-firebase/perf
\end{minted}

\subparagraph{2. Performance Monitoring tool configureren}
Daarnaast moet de plugin worden toegevoegd aan 
de dependancies binnen het \textit{/android/build.gradle} bestand. 
\begin{minted}{java}
buildscript {
    dependencies {
        // andere dependencies
        classpath "com.google.firebase:perf-plugin:1.4.2"
    }
}
\end{minted}
En moet de plugin worden uitgevoerd door deze aan het \textit{/android/app/build.gradle} bestand toe te voegen.
\begin{minted}{java}
apply plugin: "com.google.firebase.firebase-perf"
\end{minted}
Tot slot wordt deze ook toegevoegd aan de dependancies in hetzelfde \textit{/android/app/build.gradle} bestand.
\begin{minted}{java}
dependencies {
    // andere dependencies
    implementation "com.google.firebase:firebase-perf-ktx"
}
\end{minted}
Nu is het blanco project klaar om vanuit dit project alle functionaliteiten te 
implementeren en onderzoeken.

\subparagraph{3. Performance meten}
Om de performantie te meten doorheen de applicaties zal dit er als 
volgt uitzien:
\begin{minted}{typescript}
import perf from '@react-native-firebase/perf';

const trace = await perf().newTrace('naam_trace');

trace.start();
// uit te voeren code
trace.stop();

\end{minted}
De trace wordt automatisch naar de Firebase console gestuurd, van waarop de 
resultaten kunnen worden bekeken.