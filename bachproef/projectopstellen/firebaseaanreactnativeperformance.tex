\subparagraph{1. Dependancy installeren}
Als eerst moeten we de Performance Monitoring tool aan de root van ons project toevoegen. 
Dit kunnen we doen met volgend commando.
\begin{minted}{bash}
npm install --save @react-native-firebase/perf
\end{minted}

\subparagraph{2. Performance Monitoring tool configureren}
Daarnaast moeten we net zoals bij de React Native Firebase app module de plugin toevoegen aan 
de dependancies binnen het \textit{/android/build.gradle} bestand. 
\begin{minted}{java}
buildscript {
    dependencies {
        // andere dependencies
        classpath "com.google.firebase:perf-plugin:1.4.2"
    }
}
\end{minted}
En moeten we de plugin uitvoeren door deze aan het \textit{/android/app/build.gradle} bestand toe te voegen.
\begin{minted}{java}
apply plugin: "com.google.firebase.firebase-perf"
\end{minted}
En voegen we deze ook toe aan de dependancies in hetzelfde \textit{/android/app/build.gradle} bestand.
\begin{minted}{java}
dependencies {
    // andere dependancies
    implementation "com.google.firebase:firebase-perf-ktx"
}
\end{minted}
Nu is het blanco project klaar om vanuit dit project alle functionaliteiten te 
implementeren en onderzoeken.

\subparagraph{3. Performance meten}
Nu kunnen we een trace maken om de performantie te meten. Doorheen de applicaties zal dit er als 
volgt uitzien.
\begin{minted}{typescript}
import perf from '@react-native-firebase/perf';
\end{minted}
Eerst importeren we de Performance Monitoring tool. Daarna kunnen we een trace maken en starten.
\begin{minted}{typescript}
const trace = await perf().newTrace('naam_trace');

trace.start();
// uit te voeren code
trace.stop();

\end{minted}
De trace wordt gestart voor de code die we willen meten en gestopt na de code die we willen meten.
De trace wordt dan automatisch naar de Firebase console gestuurd. Hier kunnen we de trace terugvinden
en de performantie analyseren.