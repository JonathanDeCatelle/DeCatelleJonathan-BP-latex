Om bij React Native een project op te starten, wordt volgend commando gebruikt: 
\begin{minted}{bash}
npx react-native init <projectnaam>
// Of
npx react-native@X.XX.X init <projectnaam> --version X.XX.X
\end{minted}
Deze wordt uitgevoerd in de terminal van Visual Studio Code in de gewenste map waar 
het project moet worden aangemaakt.
\\\\
Het eerste commando zal een blanco React Native project aanmaken met de laatste versie. Bij het tweede commando
kan er een versie worden meegegeven door \textbf{X.XX.X} in bovenstaande commando te vervangen.
In dit onderzoek wordt er gebruik gemaakt van de laatste beschikbare versie namelijk \textbf{0.71.7}.

\subsection{Firebase Performance Monitoring}
Om de performantie van de applicaties te meten moeten de 
Firebase Performance Monitoring tool in het project worden geïmplementeerd.

\paragraph{Firebase project aanmaken}
Vooraleer de Performance Monitoring tool geïmplementeerd kan worden moet Firebase 
aan het project worden toegevoegd.

\subparagraph{1. Dependancy installeren}
Daarna moeten we de React Native Firebase "app" module aan ons React native project toevoegen. 
Dit kunnen we doen met volgend commando.
\begin{minted}{bash}
npm install --save @react-native-firebase/app
\end{minted}

\subparagraph{2. Configuratie bestanden toevoegen}
Na dat de dependancy is toegevoegd maken we in het Firebase project opnieuw een Android applicatie aan. 
Om deze aan te maken geven we de package naam dat te vinden is in \textit{android/app/build.gradle} als 
applicationId en optioneel de naam van de applicatie mee.
\\\\
Na dat de Android applicatie is aangemaakt, moeten we het \textbf{google-services.json} bestand downloaden. 
Deze plaatsen we dan in de \textit{android/app} folder.

\subparagraph{3. Firebase configureren}
Om Firebase het configuratiebestand nu te laten gebruiken moeten we de google-services plugin toevoegen aan 
de dependancies binnen het \textit{/android/build.gradle} bestand. 
\begin{minted}{java}
buildscript {
    dependencies {
        // andere dependencies
        classpath 'com.google.gms:google-services:4.3.15'
    }
}
\end{minted}
Tot slot moeten we de plugin uitvoeren door deze aan het \textit{/android/app/build.gradle} bestand toe te voegen.
\begin{minted}{java}
apply plugin: 'com.google.gms.google-services'
\end{minted}
De React Native Firebase app module is nu volledig geïmplementeerd.

\paragraph{Implementatie Performance Monitoring tool}
Nu de React Native Firebase app module is geïmplementeerd, zijn alle voorwaarden voldaan om de 
Performance Monitoring tool verder te implementeren.

\subparagraph{1. Dependancy installeren}
Als eerst moeten we de Performance Monitoring tool aan de root van ons project toevoegen. 
Dit kunnen we doen met volgend commando.
\begin{minted}{bash}
npm install --save @react-native-firebase/perf
\end{minted}

\subparagraph{2. Performance Monitoring tool configureren}
Daarnaast moeten we net zoals bij de React Native Firebase app module de plugin toevoegen aan 
de dependancies binnen het \textit{/android/build.gradle} bestand. 
\begin{minted}{java}
buildscript {
    dependencies {
        // andere dependencies
        classpath "com.google.firebase:perf-plugin:1.4.2"
    }
}
\end{minted}
En moeten we de plugin uitvoeren door deze aan het \textit{/android/app/build.gradle} bestand toe te voegen.
\begin{minted}{java}
apply plugin: "com.google.firebase.firebase-perf"
\end{minted}
En voegen we deze ook toe aan de dependancies in hetzelfde \textit{/android/app/build.gradle} bestand.
\begin{minted}{java}
dependencies {
    // andere dependancies
    implementation "com.google.firebase:firebase-perf-ktx"
}
\end{minted}
Nu is het blanco project klaar om vanuit dit project alle functionaliteiten te 
implementeren en onderzoeken.

\subparagraph{3. Performance meten}
Nu kunnen we een trace maken om de performantie te meten. Doorheen de applicaties zal dit er als 
volgt uitzien.
\begin{minted}{typescript}
import perf from '@react-native-firebase/perf';
\end{minted}
Eerst importeren we de Performance Monitoring tool. Daarna kunnen we een trace maken en starten.
\begin{minted}{typescript}
const trace = await perf().newTrace('naam_trace');

trace.start();
// uit te voeren code
trace.stop();

\end{minted}
De trace wordt gestart voor de code die we willen meten en gestopt na de code die we willen meten.
De trace wordt dan automatisch naar de Firebase console gestuurd. Hier kunnen we de trace terugvinden
en de performantie analyseren.
