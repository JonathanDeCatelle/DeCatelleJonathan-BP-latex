Om bij React Native een project op te starten, wordt volgend commando gebruikt: 
\begin{minted}{bash}
npx react-native init <projectnaam>
// Of
npx react-native@X.XX.X init <projectnaam> --version X.XX.X
\end{minted}
Deze wordt uitgevoerd in de terminal van Visual Studio Code in de gewenste map waar 
het project moet worden aangemaakt.
\\\\
Het eerste commando zal een blanco React Native project aanmaken met de laatste versie. Bij het tweede commando
kan er een versie worden meegegeven door \textbf{X.XX.X} in bovenstaande commando te vervangen.
In dit onderzoek wordt er gebruik gemaakt van de laatste beschikbare versie namelijk \textbf{0.71.7}.

\subsection{Firebase Performance Monitoring}
Om de performantie van de applicaties te meten moeten de 
Firebase Performance Monitoring tool in het project worden geïmplementeerd.

\paragraph{Firebase project aanmaken}
Vooraleer de Performance Monitoring tool geïmplementeerd kan worden moet Firebase 
aan het project worden toegevoegd.

\subparagraph{1. Dependancy installeren}
Eerst moet de React Native Firebase "app" module aan de root van het React Native project worden toegevoegd. 
Dit wordt gedaan met volgend commando:
\begin{minted}{bash}
npm install --save @react-native-firebase/app
\end{minted}

\subparagraph{2. Configuratie bestanden toevoegen}
Nadat de dependancy is toegevoegd, moet er een nieuw Firebase project worden aangemaakt. 
Om deze aan te maken moet de package naam worden meegegeven die te vinden is in het 
\textit{android/app/build.gradle} bestand als 
\textbf{applicationId}. Optioneel kan de naam van de applicatie mee worden gegeven.
\\\\
Nadat de Android applicatie is aangemaakt, moet het \textbf{google-services.json} bestand worden gedownload. 
Deze wordt dan geplaatst in de \textit{android/app} folder.

\subparagraph{3. Firebase configureren}
Om Firebase het configuratiebestand nu te laten gebruiken, moet de google-services plugin worden toegevoegd aan 
de dependancies binnen het \textit{android/build.gradle} bestand. 
\begin{minted}{java}
buildscript {
    dependencies {
        // andere dependencies
        classpath "com.google.gms:google-services:4.3.15"
    }
}
\end{minted}
Tot slot moet de plugin aan het \textit{android/app/build.gradle} bestand worden toegevoegd.
\begin{minted}{java}
apply plugin: "com.google.gms.google-services"
\end{minted}
En worden deze ook aan de dependancies toegevoegd in hetzelfde \textit{android/app/build.gradle} bestand.
\begin{minted}{java}
dependencies {
    // andere dependancies
    implementation platform("com.google.firebase:firebase-bom:32.0.0")
}
\end{minted}
De React Native Firebase app module is nu volledig geïmplementeerd.

\paragraph{Implementatie Performance Monitoring tool}
Nu de React Native Firebase app module is geïmplementeerd, zijn alle voorwaarden voldaan om de 
Performance Monitoring tool verder te implementeren.

\subparagraph{1. Dependancy installeren}
Eerst moet de Performance Monitoring tool aan de root van ons project worden toegevoegd. 
Dit wordt gedaan met volgend commando:
\begin{minted}{bash}
npm install --save @react-native-firebase/perf
\end{minted}

\subparagraph{2. Performance Monitoring tool configureren}
Daarnaast moet de plugin worden toegevoegd aan 
de dependancies binnen het \textit{/android/build.gradle} bestand. 
\begin{minted}{java}
buildscript {
    dependencies {
        // andere dependencies
        classpath "com.google.firebase:perf-plugin:1.4.2"
    }
}
\end{minted}
En moet de plugin worden uitgevoerd door deze aan het \textit{/android/app/build.gradle} bestand toe te voegen.
\begin{minted}{java}
apply plugin: "com.google.firebase.firebase-perf"
\end{minted}
Tot slot wordt deze ook toegevoegd aan de dependancies in hetzelfde \textit{/android/app/build.gradle} bestand.
\begin{minted}{java}
dependencies {
    // andere dependencies
    implementation "com.google.firebase:firebase-perf-ktx"
}
\end{minted}
Nu is het blanco project klaar om vanuit dit project alle functionaliteiten te 
implementeren en onderzoeken.

\subparagraph{3. Performance meten}
Om de performantie te meten doorheen de applicaties zal dit er als 
volgt uitzien:
\begin{minted}{typescript}
import perf from '@react-native-firebase/perf';

const trace = await perf().newTrace('naam_trace');

trace.start();
// uit te voeren code
trace.stop();

\end{minted}
De trace wordt automatisch naar de Firebase console gestuurd, van waarop de 
resultaten kunnen worden bekeken.
