Om bij React Native een project op te starten gebruiken we volgend commando. 
Deze voeren we uit in de terminal van Visual Studio Code in de gewenste map waar 
het project aangemaakt moet worden.
\begin{minted}{bash}
npx react-native init <projectnaam>
// Of
npx react-native@X.XX.X init <projectnaam> --version X.XX.X
\end{minted}

% komt bovenaan code, andere manier om caption bij code te zetten
% \begin{listing}
%     \begin{minted}{bash}
%         npx react-native init <projectnaam>
%         // Of voor een bepaalde versie Door X.XX.X te veranderen met de gewenste versie
%         npx react-native@X.XX.X init <projectnaam> --version X.XX.X
%     \end{minted}
%     \caption{Nieuw React Native project aanmaken.}
%     \label{code:reactnativeprojectopstarten}
% \end{listing}

Het eerste commando zal een blanco React Native project aanmaken met de laatste versie 
waarmee we van start kunnen gaan om de functionalieiten te implementeren. Hier kan ook 
een versie worden meegegeven, door \textbf{X.XX.X} in bovenstaande commando te vervangen 
maken we een project aan met de gespecifieerde versie. In dit onderzoek maken we gebruik 
van de laatste beschikbare versie namelijk \textbf{0.71.7}.

\subsection{Firebase Performance Monitoring}
Om de performantie de applicaties te meten moeten we net zoals bij native applicaties de 
Firebase Performance Monitoring tool in het project implementeren. Om deze te implementeren 
volgen we deze handleiding \url{https://rnfirebase.io/}.

\paragraph{Firebase project aanmaken}
Om de Firebase Performance Monitoring tool te gebruiken hebben we net zoals bij native 
als eerst een Firebase project nodig. Maar aangezien deze al is aangemaakt, moeten we dit 
niet opnieuw doen.

\subparagraph{1. Dependancy installeren}
Eerst moet de React Native Firebase "app" module aan de root van het React Native project worden toegevoegd. 
Dit wordt gedaan met volgend commando:
\begin{minted}{bash}
npm install --save @react-native-firebase/app
\end{minted}

\subparagraph{2. Configuratie bestanden toevoegen}
Nadat de dependancy is toegevoegd, moet er een nieuw Firebase project worden aangemaakt. 
Om deze aan te maken moet de package naam worden meegegeven die te vinden is in het 
\textit{android/app/build.gradle} bestand als 
\textbf{applicationId}. Optioneel kan de naam van de applicatie mee worden gegeven.
\\\\
Nadat de Android applicatie is aangemaakt, moet het \textbf{google-services.json} bestand worden gedownload. 
Deze wordt dan geplaatst in de \textit{android/app} folder.

\subparagraph{3. Firebase configureren}
Om Firebase het configuratiebestand nu te laten gebruiken, moet de google-services plugin worden toegevoegd aan 
de dependancies binnen het \textit{android/build.gradle} bestand. 
\begin{minted}{java}
buildscript {
    dependencies {
        // andere dependencies
        classpath "com.google.gms:google-services:4.3.15"
    }
}
\end{minted}
Tot slot moet de plugin aan het \textit{android/app/build.gradle} bestand worden toegevoegd.
\begin{minted}{java}
apply plugin: "com.google.gms.google-services"
\end{minted}
En worden deze ook aan de dependancies toegevoegd in hetzelfde \textit{android/app/build.gradle} bestand.
\begin{minted}{java}
dependencies {
    // andere dependancies
    implementation platform("com.google.firebase:firebase-bom:32.0.0")
}
\end{minted}
De React Native Firebase app module is nu volledig geïmplementeerd.

\paragraph{Implementatie Performance Monitoring tool}
Nu de React Native Firebase app module is geïmplementeerd zijn alle voorwaarden voldaan om de 
Performance Monitoring tool verder te implementeren. Hiervoor volgen we deze handleiding 
\url{https://rnfirebase.io/perf/usage}.

\subparagraph{1. Dependancy installeren}
Eerst moet de Performance Monitoring tool aan de root van ons project worden toegevoegd. 
Dit wordt gedaan met volgend commando:
\begin{minted}{bash}
npm install --save @react-native-firebase/perf
\end{minted}

\subparagraph{2. Performance Monitoring tool configureren}
Daarnaast moet de plugin worden toegevoegd aan 
de dependancies binnen het \textit{/android/build.gradle} bestand. 
\begin{minted}{java}
buildscript {
    dependencies {
        // andere dependencies
        classpath "com.google.firebase:perf-plugin:1.4.2"
    }
}
\end{minted}
En moet de plugin worden uitgevoerd door deze aan het \textit{/android/app/build.gradle} bestand toe te voegen.
\begin{minted}{java}
apply plugin: "com.google.firebase.firebase-perf"
\end{minted}
Tot slot wordt deze ook toegevoegd aan de dependancies in hetzelfde \textit{/android/app/build.gradle} bestand.
\begin{minted}{java}
dependencies {
    // andere dependencies
    implementation "com.google.firebase:firebase-perf-ktx"
}
\end{minted}
Nu is het blanco project klaar om vanuit dit project alle functionaliteiten te 
implementeren en onderzoeken.

\subparagraph{3. Performance meten}
Om de performantie te meten doorheen de applicaties zal dit er als 
volgt uitzien:
\begin{minted}{typescript}
import perf from '@react-native-firebase/perf';

const trace = await perf().newTrace('naam_trace');

trace.start();
// uit te voeren code
trace.stop();

\end{minted}
De trace wordt automatisch naar de Firebase console gestuurd, van waarop de 
resultaten kunnen worden bekeken.

Om de Android profiler te gebruiken, zijn er geen extra stappen nodig. Deze is standaard beschikbaar voor applicaties 
die op de emulator binnen Android Studio runnen.