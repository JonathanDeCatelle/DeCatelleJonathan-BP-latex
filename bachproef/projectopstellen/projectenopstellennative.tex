Bij Android Studio kunnen we in de IDE zelf nieuwe projecten aanmaken om de 
functionaliteiten te maken. Dit doen we met \textit{File > New > New Project\dots}. 
Hierna krijgen we een overzicht van projecten die we als basis kunnen gebruiken. Ook 
kan er op dit scherm gekozen worden voor welk apparaat de applicatie is.
\begin{figure}[H]
    \centering
    \includegraphics[height=0.35\textheight]{OverzichtProjecten.png}
    \caption{Overzicht startprojecten Android Studio.}
\end{figure}
Afhankelijk van de functionaliteit dat we gaan implementeren zullen we dus andere 
startprojecten kunnen kiezen. Na het kiezen van het startproject geven we een naam, 
programmeertaal en de minimaal ondersteunde Android API. Voor dit onderzoek kiezen we 
voor Kotlin. De naam en minimale ondersteunde Android API speelt hierbij geen rol.

\subsection{Basisfunctionaliteiten}
\label{par:basisfunctionaliteiten}
Bij de basisfunctionaliteiten zullen we als startproject 
\textbf{Bottom Navigation Views Activity} gebruiken om te kunnen navigeren tussen 
verschillende schermen. Daardoor moeten we zelf geen extra libraries gaan implementeren.

\subsection{Overige functionaliteiten}
Voor de overige functionaliteiten (gebruik van sensoren, 
push-notificaties en audio- en videospelers) zullen we als startproject 
\textbf{Empty Views Activity} gebruiken. Op die manier hebben we de minste \gls{overhead} 
dat de resultaten kan beïnvloeden.

\subsection{Firebase Performance Monitoring}
Om de performantie van de applicaties te meten, meer specifiek de tijdsduur dat een stuk code nodig heeft om uit te voeren 
gaan we de Firebase Performance Monitoring tool in het project implementeren. 

\paragraph{Firebase project aanmaken}
Om de Firebase Performance Monitoring tool te gebruiken hebben we als eerst een Firebase project nodig.
Deze kan aangemaakt worden via volgende link \url{https://console.firebase.google.com/}.

\paragraph{Firebase aan project toevoegen}
Vooraleer we de Performance Monitoring tool kunnen implementeren moeten we eerst Firebase 
aan ons project toevoegen. Hiervoor volgen we deze handleiding 
\url{https://firebase.google.com/docs/android/setup}.

\subparagraph{1. Configuratie bestanden toevoegen}
In het Firebase project voegen we een nieuwe Android applicatie toe. Om deze aan te maken geven we de package naam 
dat te vinden is in build.gradle(module) bestand als applicationId en optioneel de naam van de applicatie mee.
\\\\
Na dat de Android applicatie is aangemaakt, moeten we het \textbf{google-services.json} bestand downloaden. 
Deze plaatsen we dan in de \textit{app} folder.

\subparagraph{2. Firebase plugins configureren}
Om Firebase het configuratiebestand nu te laten gebruiken moeten we de google-services plugin toevoegen aan 
de dependancies in het build.gradle(project) bestand boven de plugins. 
\begin{minted}{java}
buildscript {
    repositories {
        google()
        mavenCentral()
    }

    dependencies {
        // andere dependancies
        classpath "com.google.gms:google-services:4.3.15"
    }
}
\end{minted}
Daarna moeten we de plugin uitvoeren door deze aan het build.gradle(module) bestand toe te voegen.
\begin{minted}{java}
plugins {
    // andere plugins 
    id "com.google.gms.google-services"
}
\end{minted}
Tot slot moeten we de Firebase SDKs toevoegen aan het build.gradle(module) bestand.
\begin{minted}{java}
dependencies {
    // andere dependancies
    implementation platform("com.google.firebase:firebase-bom:32.0.0")
}
\end{minted}
Firebase is nu volledig aan ons project toegevoegd.

\paragraph{Implementatie Performance Monitoring tool}
Nu dat Firebase aan ons project is toegevoegd kunnen we de Performance Monitor 
toevoegen door deze handleiding te volgen \url{https://firebase.google.com/docs/perf-mon/get-started-android}.

\subparagraph{1. Performance Monitoring SDK aan applicatie toevoegen}
Als eerst voegen we de Performance Monitor SDK toe aan ons project in het build.gradle(module) bestand.
\begin{minted}{java}
dependencies {
    // ... andere dependancies
    implementation "com.google.firebase:firebase-perf-ktx"
}
\end{minted}

\subparagraph{2. Performance Monitoring Gradle plugin toevoegen}
Na het toevoegen van de SDK moeten we de Perormance Monitoring Gradle plugin toevoegen. 
Dit doen we door de dependancies hiervan toe te voegen aan het build.gradle(project) bestand.
\begin{minted}{java}
dependencies {
    // ... andere dependancies
    classpath "com.google.firebase:perf-plugin:1.4.2"
}
\end{minted}
Tot slot voegen we de plugin toe. Dit doen we door deze in het build.gradle(module) bestand toe te voegen.
\begin{minted}{java}
plugins {
    // ... andere plugins
    id "com.google.firebase.firebase-perf"
}
\end{minted}
Nu is het blanco project klaar om vanuit dit project alle functionaliteiten te 
implementeren en onderzoeken, buiten de basisfunctionaliteiten. Hiervoor zullen we vanuit een ander project starten, 
maar zullen we de Perormance Monitoring tool op dezelfde manier implementeren.

\subparagraph{3. Performance meten}
Nu kunnen we een trace maken om de performantie te meten. Doorheen de applicaties zal dit er als 
volgt uitzien.
\begin{minted}{java}
val trace = FirebasePerformance.getInstance().newTrace("naam_trace")

trace.start()
// uit te voeren code
trace.stop()
\end{minted}
De trace wordt gestart voor de code die we willen meten en gestopt na de code die we willen meten.
De trace wordt dan automatisch naar de Firebase console gestuurd. Hier kunnen we de trace terugvinden
en de performantie analyseren.

Om de Android profiler te gebruiken, zijn er geen extra stappen nodig. Deze is standaard beschikbaar voor applicaties 
die op de emulator binnen Android Studio runnen.