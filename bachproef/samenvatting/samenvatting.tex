%%=============================================================================
%% Samenvatting
%%=============================================================================

% TODO: De "abstract" of samenvatting is een kernachtige (~ 1 blz. voor een
% thesis) synthese van het document.
%
% Een goede abstract biedt een kernachtig antwoord op volgende vragen:
%
% 1. Waarover gaat de bachelorproef?
% 2. Waarom heb je er over geschreven?
% 3. Hoe heb je het onderzoek uitgevoerd?
% 4. Wat waren de resultaten? Wat blijkt uit je onderzoek?
% 5. Wat betekenen je resultaten? Wat is de relevantie voor het werkveld?
%
% Daarom bestaat een abstract uit volgende componenten:
%
% - inleiding + kaderen thema
% - probleemstelling
% - (centrale) onderzoeksvraag
% - onderzoeksdoelstelling
% - methodologie
% - resultaten (beperk tot de belangrijkste, relevant voor de onderzoeksvraag)
% - conclusies, aanbevelingen, beperkingen
%
% LET OP! Een samenvatting is GEEN voorwoord!

\chapter{Samenvatting}
Bij het ontwikkelen van mobiele applicaties is de keuze tussen cross-platform of native 
ontwikkeling van cruciaal belang. Een verkeerde keuze kan een project 
maken of breken. Deze bachelorproef richt zich op het vergelijken van 
ontwikkeltijden, prestaties en schaalbaarheid van verschillende functionaliteiten tussen
native en cross-platform ontwikkeling. Het onderzoek beantwoordt de vraag: "Hoe verschillen 
de ontwikkeltijden, prestaties en schaalbaarheid van functionaliteiten tussen native en 
cross-platform ontwikkeling van mobiele applicaties?"
\\\\
Het onderzoek is uitgevoerd aan de hand van een specifiek stappenplan. Blanco projecten 
zijn aangemaakt om van daaruit de ontwikkeltijd, prestaties en schaalbaarheid van verschillende 
functionaliteiten te meten. De resultaten zijn als volgt:
\\\\
Voor basisfunctionaliteiten presteert native ontwikkeling beter dan cross-platform 
ontwikkeling, met een snellere opstarttijd en lager geheugengebruik. Beide methodes 
zijn schaalbaar en kunnen bestaande schermen hergebruiken.
\\\\
Bij audio- en videospelers presteren native applicaties over het algemeen beter dan 
cross-platform applicaties. Native applicaties hebben minder geheugengebruik en een lager 
CPU-gebruik. Beide methodes kunnen de logica abstracteren en hergebruiken.
\\\\
Bij het gebruik van sensoren presteren native applicaties gelijkaardig aan cross-platform applicaties. 
Native applicaties zijn wel sneller en hebben een lager geheugengebruik. Beide methodes zijn 
schaalbaar en kunnen sensoren gemakkelijk uitbreiden en bestaande logica hergebruiken.
\\\\
Bij het aanmaken van notificaties is er een groot verschil tussen Android en React 
Native. Android heeft een kortere creatietijd en een lager geheugengebruik. Beide methodes hebben 
vergelijkbare ontwikkeltijd en schaalbaarheid.
\\\\
Uit het onderzoek blijkt dat native ontwikkeling over het algemeen betere prestaties 
levert dan cross-platform ontwikkeling voor de onderzochte functionaliteiten. Het onderzoek 
biedt applicatie-ontwikkelaars en andere belanghebbenden inzicht in de verschillen tussen 
native en cross-platform ontwikkeling, zodat zij een weloverwogen beslissing kunnen nemen 
bij het kiezen van de ontwikkelmethode.


% \section{Inleiding}
% Bij het ontwikkelen van mobiele applicaties is de keuze tussen native of cross-platform 
% ontwikkeling van cruciaal belang. Deze keuze heeft een grote invloed op het ontwikkelproces, 
% waarbij verschillende factoren zoals tijd, budget, performantie, schaalbaarheid en gewenste 
% functionaliteiten in overweging moeten worden genomen. Native ontwikkeling maakt gebruik 
% van platform-specifieke code en speciaal ontworpen frameworks, terwijl cross-platform 
% ontwikkeling het "write once, run anywhere" principe gebruikt. Ondanks dat er al vaak vergelijkingen 
% zijn gemaakt tussen native en cross-platform ontwikkeling, is er meestal weinig aandacht 
% besteed aan het individueel vergelijken van specifieke functionaliteiten. In deze bachelorproef 
% worden er verschillende functionaliteiten vergeleken op basis van ontwikkeltijd, performantie en 
% schaalbaarheid tussen native en cross-platform ontwikkeling.

% \section{Probleemstelling}
% Voor veel bedrijven en ontwikkelaars is de keuze tussen native en cross-platform 
% ontwikkeling een complexe beslissing. Het maken van de verkeerde keuze kan een project 
% maken of breken. Native ontwikkeling kan veel tijd en geld kosten, terwijl cross-platform 
% ontwikkeling een snellere en goedkopere oplossing kan bieden. Deze bachelorproef 
% richt zich op het identificeren van de verschillen in ontwikkeltijden, prestaties en 
% schaalbaarheid van functionaliteiten tussen native en cross-platform ontwikkeling van 
% mobiele applicaties. Daarnaast wordt er ook onderzocht of er specifieke functionaliteiten 
% zijn die niet worden ondersteund door cross-platform ontwikkeling of waarbij de prestaties 
% van de functionaliteit onbruikbaar worden. 

% \section{Onderzoeksvraag}
% \begin{itemize}
%     \item Hoe verschillen de ontwikkeltijden, prestaties en schaalbaarheid van functionaliteiten tussen native en cross-platform ontwikkeling van mobiele applicaties?
% \end{itemize}
% Deze onderzoeksvraag richt zich op het vergelijken van ontwikkeltijden, prestaties 
% en schaalbaarheid van functionaliteiten, met als doel inzicht te krijgen in de verschillen 
% tussen native en cross-platform ontwikkeling. Met de resultaten van dit onderzoek 
% kunnen applicatie-ontwikkelaars en andere belanghebbenden beter geïnformeerde 
% beslissingen nemen bij het kiezen van de meest geschikte ontwikkelmethode voor hun 
% specifieke behoeften.

% \section{Onderzoeksdoelstelling}
% Het uiteindelijke doel van dit onderzoek is om 
% applicatie-ontwikkelaars, bedrijven en andere geïnteresseerden een beter inzicht te geven 
% in de verschillen tussen native en cross-platform ontwikkeling, zodat ze een weloverwogen 
% beslissing kunnen nemen bij het kiezen van de ontwikkelmethode. Bovendien zal dit onderzoek 
% ook bijdragen aan een beter begrip van ontwikkeltijd, performantie en schaalbaarheid van 
% functionaliteiten in het algemeen.

% \section{Methodologie}
% Het onderzoek volgt een specifiek stappenplan om de functionaliteiten te onderzoeken. 
% Dit omvat het aanmaken van blanco projecten, het meten van ontwikkeltijd, het meten van 
% de performantie en het onderzoeken van de schaalbaarheid. Elk van de functionaliteiten, 
% waaronder basisfunctionaliteiten, audio- en videospelers, sensoren en push-notificaties, 
% wordt afzonderlijk behandeld in hoofdstukken 6 tot 9. De ontwikkeltijd van elke 
% functionaliteit wordt gedocumenteerd, inclusief eventuele bugs of problemen. Daarnaast 
% wordt de performantie van de functionaliteiten gemeten, waarbij zowel de Android Profiler 
% als de Firebase Performance Monitoring tool worden gebruikt. De complexiteit en 
% herbruikbaarheid van de code worden geanalyseerd om inzicht te krijgen in de schaalbaarheid 
% van de functionaliteiten. De resultaten en individuele conclusies van de functionaliteiten worden 
% samengevat in hoofdstuk 10 om een antwoord te geven op de onderzoeksvragen uit hoofdstuk 1.

% \section{Resultaten}
% \subsection{Basisfunctionaliteiten}
% Native ontwikkeling presteert beter dan cross-platform ontwikkeling voor basisfunctionaliteiten, 
% met een snellere opstarttijd (2,58s vs. 2,84s) en lager geheugengebruik (gemiddeld 81MB vs. 202MB). 
% Beide applicaties tonen vergelijkbaar CPU-gebruik, maar native applicaties scoren beter op snelheid 
% en geheugen. Native ontwikkeling met Android Studio is sneller voor basisfunctionaliteiten, terwijl 
% cross-platform ontwikkeling meer tijd vergt door installatieprocessen en navigatieopbouw. Beide 
% methodes zijn schaalbaar en kunnen bestaande schermen hergebruiken.

% \subsection{Audio- en videospelers}
% Uit het onderzoek blijkt dat native applicaties over het algemeen betere prestaties 
% leveren dan cross-platform applicaties. Native applicaties hebben minder 
% geheugengebruik (73MB vs. 207MB) en een lager CPU-gebruik tijdens het afspelen van audio en video.
% Daarnaast is de ontwikkeltijd voor native applicaties lager, maar cross-platform biedt extra 
% functionaliteit, zoals het afspelen van audio en video in de achtergrond. Beide methodes kunnen 
% de logica abstracteren en hergebruiken.

% \subsection{Gebruik van sensoren}
% Native applicaties presteren beter dan cross-platform applicaties bij het ophalen van 
% sensorgegevens. Ze zijn sneller omdat de data rechtstreeks naar de applicatie wordt gestuurd, 
% terwijl cross-platform data naar een wrapper en vervolgens naar de applicatie moet sturen. 
% Het geheugengebruik is vergelijkbaar, maar cross-platform vereist gemiddeld 121,59\% meer 
% geheugen vanwege de extra library-implementatie. Bij het CPU-gebruik  
% voor de accelerometer, heeft native een constant gemiddeld gebruik van 23\% vergeleken met 
% een piek van 50\% gevolgd door 9\% bij cross-platform. Bij de gyroscoop is het verschil 
% vergelijkbaar. Ook de ontwikkeltijd is vergelijkbaar, maar cross-platform kan langer 
% duren vanwege de extra library-implementatie. Beide methodes zijn schaalbaar en kunnen 
% sensoren gemakkelijk uitbreiden en bestaande logica hergebruiken.

% \subsection{Notificaties}
% Bij het aanmaken van notificaties is er een significant verschil tussen Android en 
% React Native. Gemiddeld duurt het creëren van een notificatie bij Android 154 
% ms minder lang dan bij React Native. Hoewel er geen verschil is in geheugengebruik 
% bij het aanmaken van notificaties, gebruikt React Native 112,35\% meer geheugen 
% dan Android. Het CPU-gebruik verschilt ook, waarbij React Native gemiddeld een 
% piek van 28\% heeft, terwijl Android op 21\% blijft. De ontwikkeltijd en schaalbaarheid 
% zijn vergelijkbaar voor beide methodes.

% \section{Conclusies}
% Dit onderzoek richtte zich op het vergelijken van prestaties, schaalbaarheid en 
% ontwikkeltijd tussen native en cross-platform ontwikkeling van mobiele applicaties. 
% De ontwikkeltijd bleek over het algemeen langer te zijn bij cross-platform vanwege de 
% extra installatiestappen en potentieel meer fouten. De compiletijd was ook langer bij 
% cross-platform, maar het voordeel van hot reload in React Native verminderde de noodzaak 
% van frequente compilatie. Wat betreft de prestaties waren er enkele verschillen, zoals 
% een aanzienlijke langere tijdsduur voor het aanmaken van notificaties bij cross-platform. 
% Het CPU-gebruik varieerde afhankelijk van de functionaliteit, terwijl er een aanzienlijk 
% hoger gemiddeld geheugengebruik was bij cross-platform. Wat schaalbaarheid betreft, waren 
% zowel native als cross-platform geschikt voor codehergebruik en opschaling. 
