%%=============================================================================
%% Gebruik van sensoren
%%=============================================================================

\chapter{Gebruik van sensoren}%
\label{ch:sensoren}

In dit hoofdstuk gaan we het gebruik van sensoren bij native en cross-platform vergelijken. 
Met deze resultaten kunnen we dan een gepaste conclusie vormen.

\section{Native}
\subsubsection{Wat hebben we nodig}
Om toegang te krijgen tot de sensoren bij native ontwikkeling moeten we geen extra library of tools installeren.
Android studio bied een aantal klassen aan die we kunnen gebruiken. Deze klassen zijn SensorManager, 
SensorEventListener en SensorEvent. Dankzij deze klassen kunnen we de data van sensoren zoals accelerometer en 
gyroscoop opvragen.

\subsubsection{Uitvoering}

\paragraph{1. Permissions toevoegen}
Om toegang te krijgen tot de sensoren moet er een user-permission toegevoegd worden aan het 
AndroidManifest.xml bestand. Deze permission gaat over de algemene betrekking tot de sensoren.
\begin{minted}{xml}
<uses-permission android:name="android.permission.ACCESS_FINE_LOCATION" />
\end{minted}

\paragraph{2. SensorManager initialiseren}
Om toegang te krijgen tot de sensoren moet er een instantie van de SensorManager klasse aangemaakt worden.
\begin{minted}{kotlin}
private lateinit var sensorManager: SensorManager
private var accelerometer: Sensor? = null
private var gyroscope: Sensor? = null
\end{minted}
Daarna worden de variabelen met behulp van de sensorManager gelinkt aan een sensor met de onCreate methode.
\begin{minted}{kotlin}
override fun onCreate(savedInstanceState: Bundle?) {
    super.onCreate(savedInstanceState)
    setContentView(R.layout.activity_main)

    sensorManager = getSystemService(Context.SENSOR_SERVICE) as SensorManager
    accelerometer = sensorManager.getDefaultSensor(Sensor.TYPE_ACCELEROMETER)
    gyroscope = sensorManager.getDefaultSensor(Sensor.TYPE_GYROSCOPE)
}
\end{minted}

\paragraph{3. SensorEventListener initialiseren}
Daarna de SensorEventListener worden aangemaakt hiervoor wordt de onSensorChanged methode gebruikt.
\begin{minted}{kotlin}
private val sensorEventListener = object : SensorEventListener {
    override fun onSensorChanged(event: SensorEvent) {
        if (event.sensor.type == Sensor.TYPE_ACCELEROMETER) {
            val x = event.values[0]
            val y = event.values[1]
            val z = event.values[2]
            // Doe iets met de accelerometerwaarden (x, y, z)
        } else if (event.sensor.type == Sensor.TYPE_GYROSCOPE) {
            val x = event.values[0]
            val y = event.values[1]
            val z = event.values[2]
            // Doe iets met de gyroscoopwaarden (x, y, z)
        }
    }

    override fun onAccuracyChanged(sensor: Sensor?, accuracy: Int) {
        return // Niet nodig voor deze demo
    }
}
\end{minted}
Nu kan de data van de sensoren worden uitgelezen in de onSensorChanged methode.
\begin{minted}{kotlin}
fetchButton.setOnClickListener {
    sensorManager.registerListener(
        sensorEventListener,
        sensor, // Veranderen door gyroscope of accelerometer
        SensorManager.SENSOR_DELAY_NORMAL
    )
}
\end{minted}

\paragraph{4. Applicatie maken}
Met deze informatie wordt een applicatie ogebouwd die de data van sensoren 
ophaald. De applicatie bestaat uit twee \textbf{TextView} componenten voor de data van de 
accelerometer en gyroscoop en tot slot twee \textbf{Button} componenten om de data op te halen. 
Als de knoppen ingedrukt worden, dan worden de setOnClickListener methodes aangeroepen.
In de onSensorChanged methode wordt de data van de sensoren opgehaald en in de TextViews
geplaatst.
\begin{figure}[H]
    \centering
    \includegraphics[height=0.5\textheight]{sensoren_layoutnative.png}
    \caption{Layout van applicatie voor data van sensoren op te halen bij Android.}
\end{figure}


\subsubsection{Ontwikkeltijd}

Aangezien dat er geen extra libraries of tools gebruikt worden om de sensoren te gebruiken, 
kunnen de sensoren snel geïmplementeerd worden. Enkel moeten de juiste 
permission worden toegevoegd aan het AndroidManifest.xml bestand, de juiste klasse moet geïmporteerd worden en de juiste 
methode moet worden aangeroepen. Daarom is de tijd nodig om de sensoren te gebruiken dus zeer laag. 
Er is ongeveer 45 minuten gespendeerd om de sensoren te implementeren, inclusief opzoekwerk. Er zijn ook geen grote problemen of 
bugs voorgekomen tijdens de implementatie.




\subsubsection{Performantie}

\paragraph{Tijdsduur}
\begin{figure}[H]
    \centering
    \includegraphics[height=0.1\textheight]{sensorenDuratieNativeAccelerometer.png}
    \caption{Overzicht tijdsduur ophalen van accelerometer data bij Android.}
\end{figure}
Vanaf dat we de knop drukken om de gegevens op te halen van de accelerometer blijft de applicatie
dit continu doen. Hierdoor worden er honderden metingen gedaan die ons vertellen dat het ophalen 
van de accelerometer data gemiddel 2ms duurt. De minimum en maximum waarden liggen op 736µs en 7ms.
\begin{figure}[H]
    \centering
    \includegraphics[height=0.1\textheight]{sensorenDuratieNativeGyroscoop.png}
    \caption{Overzicht tijdsduur ophalen van accelerometer data bij Android.}
\end{figure}
Net zoals bij de accelerometer worden er constant gegevens opgehaald vanaf dat we de knop drukken om de 
gegevens op te halen. Hierdoor worden er opnieuw honderden metingen gedaan die ons vertellen dat het ophalen 
van de gyroscoop data gemiddel 2ms duurt. De minimum en maximum waarden liggen op 996µs en 3ms.

\paragraph{CPU \& geheugen}
\begin{figure}[H]
    \centering
    \includegraphics[height=0.3\textheight]{sensorenPerformantieNativeAccelerometer.png}
    \caption{Overzicht CPU en geheugen gebruik tijdens het ophalen van accelerometer data bij Android.}
\end{figure}
Op de grafiek kunnen we zien dat het CPU gebruik van de applicatie bij het ophalen van de accelerometer data,
gemiddeld 23\% is met toch wel redelijke schommelingen. Wanneer er nog geen data wordt opgehaald wordt de CPU niet 
gebruikt. Het geheugen blijft rond de 88MB hangen, met verschillen van maximum 4-5MB. 
Er is geen merkbaar verschil in het geheugen wanneer er data wordt opgehaald of wanneer er 
geen data wordt opgehaald.
\begin{figure}[H]
    \centering
    \includegraphics[height=0.3\textheight]{sensorenPerformantieNativeGyroscoop.png}
    \caption{Overzicht CPU en geheugen gebruik tijdens het ophalen van gyroscoop data bij Android.}
\end{figure}
Net zoals bij de accelerometer kunnen we op de grafiek zien dat het CPU gebruik van de applicatie bij het 
ophalen van de gyroscoop data, gemiddeld 8\% is. Wat opvalt is dat er bij het ophalen van de gyroscoop data
in het begin een piek is van 30\%. Maar dat dit daarna terug zakt naar 8\%. Bij de accelerometer was er geen
piek te zien. maar was het gemiddelde CPU gebruik wel hoger. Net zoals bij de accelerometer is er ook geen 
CPU gebruik wanneer er geen data wordt opgehaald. Het geheugen blijft terug rond de 88MB hangen, met 
verschillen van maximum 4-5MB. Ook is er opnieuw geen merkbaar verschil in het 
geheugen wanneer er data wordt opgehaald of wanneer er geen data wordt opgehaald.
  

\subsubsection{Schaalbaarheid}

\paragraph{Complexiteit}
Het proces om de data van sensoren te verkrijgen is vrij simpel. De ontwikkelaar moet enkel de juiste klasse importeren, 
de juiste methode aanroepen en de juiste permissies toevoegen aan de AndroidManifest.xml file.

\paragraph{Herbruikbaarheid}
Het is gemakkelijk om de gegevens van de sensoren te hergebruiken. De gegevens kunnen bijvoorbeeld 
worden opgeslagen in een object in plaats van te tonen in een TextView component. Dat object kan dan vanuit 
andere klasses worden opgevraagd. Het is ook mogelijk om de logica voor het ophalen van de gegevens
in een aparte klasse te plaatsen. Zo kan de logica gemakkelijk worden hergebruikt in andere klassen en 
kan deze ook aangepast of opgeschaald worden.



\section{Cross-platform}
\subsubsection{Wat hebben we nodig}
Om bij React native toegang te krijgen tot de sensoren moeten we gebruik maken van de react-native-sensors library.
Deze library biedt onder andere toegang tot de accelerometer en gyroscoop. Deze library is een wrapper voor de
native sensoren van Android en iOS. 

\subsubsection{Uitvoering}

\paragraph{1. Library toevoegen}
Eerst moet de React Native Sensors library worden toegevoegd aan de root van ons project.
Deze wordt toegevoegd met volgende commando:
\begin{minted}{bash}
npm install react-native-sensors --save
\end{minted}

\paragraph{2. Package teruggeven}
Normaal gezien moet de package dan worden toegevoegd aan het 
\textit{android/app/src/} \textbf{main/java/com/project/MainApplication.java} bestand.
Maar dit is niet meer nodig bij React Native 0.60+.

\paragraph{3. gradle instellingen aanpassen}
Tot slot moet volgende regel worden toegevoegd aan het \textit{android/app/build.gradle} bestand.
\begin{minted}{groovy}
implementation project(':react-native-sensors')
\end{minted}
En moet volgende regel worden toegevoegd aan het \textit{android/settings.gradle} bestand.
\begin{minted}{groovy}
include ':react-native-sensors'
project(':react-native-sensors').projectDir = 
    new File(rootProject.projectDir
        , '../node_modules/react-native-sensors/android')
\end{minted}
De package is nu volledig geïnstalleerd en klaar voor gebruik.

\paragraph{4. Sensor gebruiken}
Eerst wordt de library in het bestand waar we deze nodig hebben geïmporteerd.
\begin{minted}{typescript}
import { accelerometer, gyroscope } from "react-native-sensors";
\end{minted}
Daarna worden twee variabelen die de data van de sensoren zal bewaren aangemaakt.
\begin{minted}{typescript}
const [accelerometerData, setAccelerometerData] = useState({});
const [gyroscopeData, setGyroscopeData] = useState({});
\end{minted}
Tot slot kunnen deze variabelen gebruikt worden om de data van de sensoren op te vangen.
\begin{minted}{typescript}
const getData = () => {
    setIsFetchingData(true);

    startFetchingData();
};

const startFetchingData = () => {
    if (isFetchingData) {
        const accelerometerSubscription = new Accelerometer({
            updateInterval: 100, // Verander dit indien nodig
        }).subscribe(({ x, y, z }) => {
            // Doe iets met de accelerometerwaarden (x, y, z)
        });

        const gyroscopeSubscription = new Gyroscope({
        updateInterval: 100, // Verander dit indien nodig
        }).subscribe(({ x, y, z }) => {
            // Doe iets met de accelerometerwaarden (x, y, z)
        });

        // Unsubscribe van de sensoren wanneer je klaar bent 
        // met het ophalen van de data
        return () => {
            accelerometerSubscription.unsubscribe();
            gyroscopeSubscription.unsubscribe();
        };
    }
};
\end{minted}

\paragraph{4. Applicatie maken}
Net zoals bij de native applicatie wordt een applicatie gemaakt die de accelerometer en gyroscoop data 
opvraagt en weergeeft. Deze bestaat uit twee \textbf{<Text>} componenten voor de data van de
accelerometer en gyroscoop weer te geven en tot slot twee \textbf{<Button>} componenten om de 
data op te halen. Als de knoppen ingedrukt worden, dan wordt ofwel de
\textbf{getAccelerometerData} of \textbf{getGyroscopeData} methode aangeroepen. In deze methodes wordt de data 
van de sensoren opgehaald en in de \textbf{<Text>} componenten geplaatst.
\begin{figure}[H]
    \centering
    \includegraphics[height=0.4\textheight]{sensoren_layoutcross.png}
    \caption{Layout van applicatie voor data van sensoren op te halen bij React Native.}
\end{figure}


\subsubsection{Ontwikkeltijd}

In vergelijking met native ontwikkeling, moeten er wel meer stappen ondernomen worden om de 
sensoren te gebruiken. Eerst en vooral moet er een externe library geïmplementeerd worden 
om de sensoren te gebruiken. Pas daarna kunnen de sensoren worden gebruikt. Daardoor is de tijd die nodig
was om de sensoren te gebruiken hoger dan bij native ontwikkeling. We hebben ongeveer
1 uur en 15 minuten gespendeerd om de sensoren te gebruiken. Er zijn ook geen grote problemen of
bugs voorgekomen tijdens de implementatie.


\subsubsection{Performantie}

\paragraph{Tijdsduur}
\begin{figure}[H]
    \centering
    \includegraphics[height=0.1\textheight]{sensorenDuratieCrossAccelerometer.png}
    \caption{Overzicht tijdsduur ophalen van accelerometer data bij React Native.}
\end{figure}
Net zoals bij native worden er constant gegevens opgehaald vanaf dat we de knop drukken om de
gegevens op te halen. Hierdoor worden er opnieuw honderden metingen gedaan die ons vertellen dat het ophalen
van de accelerometer data gemiddel 12ms duurt. De minimum en maximum waarden liggen op 310µs en 143ms.
Wat wel opvlat bij React Native is dat de eerste meting altijd langer duurt dan de rest. 
Hierdoor is het maximum veel hoger dan bij native. Bij native is de eerste meting even snel als de rest. 
\begin{figure}[H]
    \centering
    \includegraphics[height=0.1\textheight]{sensorenDuratieCrossGyroscoop.png}
    \caption{Overzicht tijdsduur ophalen van gyroscoop data bij React Native.}
\end{figure}
Net zoals bij de accelerometer worden er constant gegevens opgehaald vanaf dat we de knop drukken om de 
gegevens op te halen. Hierdoor worden er opnieuw honderden metingen gedaan die ons vertellen dat het ophalen
van de gyroscoop data gemiddel 11ms duurt. De minimum en maximum waarden liggen op 203µs en 163ms.
Maar ook opnieuw zoals bij de accelerometer is de eerste meting veel trager dan de rest.

\paragraph{CPU \& geheugen}
\begin{figure}[H]
    \centering
    \includegraphics[height=0.25\textheight]{sensorenPerformantieCrossAccelerometer.png}
    \caption{Overzicht CPU en geheugen gebruik tijdens het ophalen van accelerometer data bij React Native.}
\end{figure}
Op de grafiek kunnen we zien dat het CPU gebruik van de applicatie bij het ophalen van de accelerometer data,
piekt tot net geen 50\% en gemiddeld 9\% is na de piek. Wat ook opvalt is dat het CPU gebruik tijdens het ophalen 
van de data schommelt tussen 0\% en 20\%. Het geheugen gebruik blijft stabiel rond de 192MB hangen, met verschillen van 
maximum 3-4MB. Ook is er opnieuw geen merkbaar verschil in het 
geheugen wanneer er data wordt opgehaald of wanneer er geen data wordt opgehaald.
\begin{figure}[H]
    \centering
    \includegraphics[height=0.25\textheight]{sensorenPerformantieCrossGyroscoop.png}
    \caption{Overzicht CPU en geheugen gebruik tijdens het ophalen van gyroscoop data bij React Native.}
\end{figure}
Net zoals bij native is er een piek in het CPU gebruik bij het starten van het ophalen van de gyroscoop data. 
De piek ligt hier wel hoger dan bij native, namelijk 40\%. Na de piek ligt het gemiddeld CPU gebruik ligt rond de 10\%. 
En net zoals bij de accelerometer schommelt het CPU gebruik tussen 0\% en 20\%. Het geheugen gebruik blijft terug 
stabiel rond de 197MB hangen, met verschillen van maximum 3-4MB. Net zoals bij de accelerometer is er  
geen merkbaar verschil in het geheugen wanneer er data wordt opgehaald of wanneer er geen 
data wordt opgehaald.

\subsubsection{Schaalbaarheid}

\paragraph{Complexiteit}
Het gebruiken van de sensoren is net zoals bij native vrij simpel. Ondanks dat er een extra library moet worden
geïmplementeerd. De package zorgt ervoor dat het ophalen van de data van de sensoren overeenkomt met het 
ophalen van de data van de sensoren bij native. Daarnaast is er ook net zoals bij native de documentatie van
de library die alles duidelijk uitlegt.

\paragraph{Herbruikbaarheid}
Net zoals bij native is het gemakkelijk om de gegevens van de sensoren te hergebruiken. We kunnen bijvoorbeeld in plaats van de
gegevens in een useState hook te plaatsen, de gegevens in een globale variabel of in een useContext opslaan. 
Om dan in andere componenten de gegevens op te vragen. Het is ook mogelijk om de logica voor het ophalen van de gegevens
in een aparte useContext te plaatsen. Zo kan de logica gemakkelijk worden hergebruikt in andere componenten.
En kan deze ook aangepast of opgeschaald worden. 



\section{Conclusie}






















