%---------- Inleiding ---------------------------------------------------------

\section{Introductie}%
\label{sec:introductie}

In dit onderzoek zullen we nagaan vanaf wanneer het interessanter is om native te gaan ontwikkelen in 
plaats van cross-platform. Hiervoor gaan we kijken naar de performantie, functionaliteit, schaalbaarheid 
en kost tussen native en cross-platform applicaties.
\\\\
Bij native applicatieontwikkeling zullen we 2 ontwikkelingsmethodes kiezen en voor cross-platform 
1 ontwikkelingsmethode gebaseerd op hun populariteit. Dit gaan we doen om het onderzoek te baseren 
op de meest gebruikte en relevante ontwikkelingsmethode. Voor native gebruiken we 2 
ontwikkelingsmethodes om zo ook het verschil tussen Android en IOS in kaart te brengen. 
Voor cross-platform zullen we 1 ontwikkelingsmethode gebruiken aangezien deze over allebei 
de platformen zal werken. We zullen meerdere applicaties gebruiken om zo ook rekening te houden 
met de grote en complexiteit van een applicatie.
\\\\
Hierdoor kunnen we kleine tot middelgrote organisaties of bedrijven helpen met het kiezen van 
een gebruikte ontwikkelingsmethode.

%---------- Stand van zaken ---------------------------------------------------

\section{State-of-the-art}%
\label{sec:state-of-the-art}

\subsection{Keuze ontwikkelingssoftware}
Om te bepalen met welke software dat we ons onderzoek gaan uitvoeren zullen we een van de 
meest populaire en relevante ontwikkelingsmethode kiezen. 
\subsubsection{Android}
Android applicaties worden allemaal ontwikkeld in Java of Kotlin \autocite{Faisal2021}. 
Daarnaast moeten we ook kijken naar welke IDE we kunnen gebruiken om in Java of Kotlin te 
ontwikkelen. Hier hebben we keuze uit tientallen verschillende IDE's namelijk: Eclipse, 
Visual Studio, Android Studio, IntelliJ, NetBeans, Komodo, Cordova en PhoneGap \autocite{Harvey2021}. 
\\\\
We zullen kiezen voor Android Studio aangezien dat dit de officiële IDE is voor Android 
applicaties. \autocite{JavaTpoint2019}
\subsubsection{Android Studio}
Android Studio is een IDE dat op 1 mei 2013 ontwikkeld is. Het wordt door Google gebruikt 
als hun officiële IDE en het heeft heel wat functionaliteiten dat het ontwikkelen van Android 
applicaties ondersteund. In Android Studio maken ze gebruik van Kotlin voor hun applicaties te 
maken. \autocite{JavaTpoint2019}

\subsubsection{IOS}
Ondanks dat Android meer gebruikt wordt dan IOS zijn er heel wat programmeertalen die gebruikt 
kunnen worden voor het ontwikkelen van IOS applicaties. We hebben keuzen uit: Objective-C, Swift, 
C++, Python \autocite{Sahu2020}. 
\\\\
Objective-C was de meest gebruikte tot 2014. Het is een object -georiënteerde programmeertaal 
met heel wat functies. Swift is in 2014 op de markt gebracht door Apple als het beste van het 
best en biedt heel van voordelen bij het ontwikkelen van applicaties. C++ wordt gemixt met Objective-C 
om IOS applicaties te maken en Python kan gebruikt worden om IOS applicaties te schrijven, maar 
het moet daarna omgezet worden naar Objective-C. \autocite{Sahu2020} 
\\\\
Naast de programmeertaal zullen we opnieuw kijken naar welke IDE we zullen gebruiken. Hiervoor 
kiezen we voor Xcode aangezien dat deze door Apple gemaakt is om IOS applicaties te maken. 
\autocite{TechCommuters2020} 
\subsubsection{Xcode}
Zoals net gezegd, Xcode is de officiële tool voor IOS applicaties te maken, ontwikkeld door 
Apple. Het maakt gebruik van Swift en je kan er applicaties mee ontwikkelen voor Iphone, Ipad, 
AppleTV of zelf MAC applicaties. Het wordt ook gezien als de meest krachtige en betrouwbare tool 
nu op de markt\autocite{TechCommuters2020}. Daarom zullen we deze gebruiken in ons onderzoek.

\subsubsection{Cross-platform}
Voor onze cross-platform ontwikkelingsmethode gaan we terug kijken naar een populair en 
relevante framework. We hebben keuze uit: Xamarin, Codename One, Flutter, React native en 
NativeScript \autocite{Zubair2022}. Elke van deze frameworks is een open-source cross-platform 
framework waarmee je applicaties kan maken. Xamarin gebruikt C\# en het .Net framework om apps 
te maken. Het wordt gebruikt door UPS en Microsoft. Codename One gebruikt Java of Kotlin en wordt 
gebruikt door Muving en HBZ. Flutter, dat gemaakt is door Google gebruikt Dart en wordt gebruikt 
door Google, eBay, Alibaba en BMW. React native is gemaakt door Facebook en het gebruikt Javascript 
en React.js om applicaties te maken. Het wordt onder andere gebruikt door Facebook, Bloomberg, 
Walmart, Uber en Shopify. NativeScript gebruikt Javascript, TypeScript, Angular of Vue.js en wordt 
gebruikt door Strudel en BitPoints. \autocite{Thaker2022}
\\\\
Voor onze cross-platform ontwikkelingsmethode gaan we kiezen voor React native. We doen dit 
omdat React native een van de meest populaire en ondersteunde frameworks is. Daarnaast maakt 
React native ook gebruik van Javascript, een van de meest populaire programmeertalen.

\subsubsection{Overzicht keuze}
We zullen voor native ontwikkelen bij Android gebruik maken van Kotlin in Android Studio en voor 
IOS van Swift in Xcode. Voor cross-platform ontwikkelen gaan we gebruik maken van Javascript in React native.

%---------- Methodologie ------------------------------------------------------
\section{Methodologie}%
\label{sec:methodologie}

We zullen in het onderzoek de geselecteerde ontwikkelingsmethodes af toetsen op basis van drie 
categorieën die belangrijk zijn bij het gebruik ervan. Zo gaan we kijken of dat native 
ontwikkelingsmethodes beter zijn dan cross-platform en indien ze beter zijn zullen we kijken vanaf 
wanneer het interessanter is om native te ontwikkelen in plaats van cross-platform op basis van 
de grote en de complexiteit van een applicatie. De drie categorieën bestaan uit:
\begin{enumerate}
  \item Hoe snel start de applicatie op en hoe responsive is hij? (Performantie)
  \item Hoe gemakkelijk kunnen we ons applicatie uitbreiden zonder veel code te moeten wijzigen en/of toevoegen? (Schaalbaarheid)
  \item Laat cross-platform even veel functionaliteiten toe als native? (Functionaliteit)
\end{enumerate}
Na het testen van elke ontwikkelingsmethode op elk van deze categorieën zullen we een besluit 
kunnen nemen of dat het interessanter is om native te ontwikkelen in plaats van cross-platform. 
Dit zullen we doen door te kijken of dat een eventueel verschil van performantie, schaalbaarheid 
of functionaliteit het waard is om te investeren in native ontwikkeling in plaats van 
cross-platform afhankelijk van de grootte en complexiteit van de applicatie.

\subsection{Performantie}
De performantie van een framework is van groot belang binnen applicaties, daarom zullen 
we deze categorie als eerst gebruiken om de ontwikkelingsmethodes op te testen.We zullen 
elk ontwikkelingsmethode testen op hun opstartsnelheid en eventuele connectie met een 
database (ophalen van data). We zullen dan alle applicaties laten verbinden met dezelfde 
database, waardoor de performantie zal getest worden.

\subsection{Schaalbaarheid}
De schaalbaarheid van een ontwikkelingsmethode is niet voor elke programmeur of applicatie 
van zeer groot belang, maar in dit onderzoek is het wel een zeer belangrijk punt. Als je 
native zou werken moet je namelijk twee keer de applicatie gaan uitbreiden en bij cross-platform 
maar één keer. We zullen dus kijken hoe makkelijk functionaliteiten bij een bestaande applicatie 
kunnen toevoegen of integreren, zonder dat er al te veel code moet aangepast worden.

\subsection{Functionaliteit}
Tot slot zullen we onderzoeken of dat cross-platform alle functionaliteiten of toch de 
minimaal nodige functionaliteiten ondersteund dat een applicatie nodig heeft. We zullen 
dus kijken vanaf welke complexiteit dat de cross-platform ontwikkelingsmethode niet meer 
alle functionaliteiten heeft of ondersteund dat native wel ondersteund.

%---------- Verwachte resultaten ----------------------------------------------
\section{Verwacht resultaat, conclusie}%
\label{sec:verwachte_resultaten}

Er wordt verwacht dat het beter is om simpele en kleine applicaties die niet veel 
functionaliteiten nodig hebben te maken met cross-platform ontwikkelingsmethodes. 
Daarnaast verwachten we dat naarmate een applicatie groter en complexer wordt het interessanter 
is om te investeren in native ontwikkelingsmethodes om zo volledig gebruik te kunnen maken van 
alle functionaliteiten. Ook omdat deze efficiënter zullen zijn aangezien dat ze volledig gebruik 
kunnen maken van het apparaat waarop de applicatie draait.

